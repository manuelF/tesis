El software que servirá como caso de estudio para este trabajo es un software de 
din\'amica molecular denominado LIO. Este software ha sido objeto de otros 
trabajos de investigación~\cite{PaperNitscheManu}~\cite{TesisNitsche} y ha sido 
usado en trabajos de simulaci\'on en din\'amica molecular.

A continuaci\'on presentaremos un breve res\'umen de los aspectos te\'oricos detr\'as de
la herramienta, de manera de servir de base para entender las distintas partes optimizadas
del mismo.

El prop\'osito de este \textit{software} es realizar simulaciones de las propiedaes
de sistemas qu\'imicos complejos (solutos en soluci\'on, prote\'inas, etc.). Estas
propiedades est\'an estudiadas dentro del marco te\'orico de la mec\'anica cu\'antica
modelando el sistema como un conjunto de part\'iculas.

En principio todas las propiedades de un sistema de part\'iculas livianas
(como los electr\'ones) pueden ser obtenidas a partir de la funci\'on de onda $\Psi$,
la cual obedece la ecuaci\'on de onda de Schr\"odinger \textit{dependiente del tiempo},

\begin{equation}
    \label{schro_time_dep}
    -\hbar\frac{\partial \Psi}{\partial t} (\mathbf{r},t) = \frac{-\hbar^2}{2\mu}\nabla^2 \Psi(\mathbf{r},t) + V(\mathbf{r},t) \Psi(\mathbf{r},t)
\end{equation}

donde $\mathbf{r} = (r_1,\dots,r_n)$ es el vector de todas las posiciones de las part\'iculas del sistema,
$m$ es la masa de una part\'icula cualquiera, $V$ es un campo externo que afecta a las part\'iculas y
$\hbar$ es la constante de Planck divida por $2\pi$. En esta versi\'on, el campo $V$ depende del tiempo; si
esto no ocurre se puede simplificar utilizando el operador Hamiltoniano

\begin{equation*}
    \hat{H} =  -\frac{\hbar^2}{m} \nabla^2 + \hat{V}
\end{equation*}

a la ecuaci\'on de Sch\"odinger \textit{independiente del tiempo}

\begin{equation}
    \label{schro_time_indep}
    \hat{H} \Psi(\mathbf{r}) = E \Psi(\mathbf{r})
\end{equation}

Donde $E$ es la energ\'ia asociada a la funci\'on de onda $\Psi$.

Ahora, resolver esta ecuaci\'on diferencial no se puede hacer de manera exacta para sistemas que involucran m\'as
de una part\'icula, por lo cual se utilizan m\'etodos aproximados. Estos m\'etodos son computacionalmente muy caros
(del \'orden c\'ubico o mayor) por lo cual otro enfoque es necesario para resolver el problema.

El enfoque de LIO esta dentro del marco de t\'ecnicas de simulaci\'on h\'ibrida denominadas QM/MM 
(\textit{Quantum Mechanical}/\textit{Molecular Mechanical}). Este enfoque consiste en resolver la ecuaci\'on de
Schr\"odinger en la porci\'on activa del sistema (por ejemplo el soluto) y modelar el resto (el solvente por
ejemplo) mediante din\'amica cl\'asica. De esta manera se obtiene, con un costo moderado, el comportamiento de
la porci\'on activa modulado por el comportamiento del resto del sistema.

Para resolver num\'ericamente~\ref{schro_time_indep} en la porci\'on activa del sistema se utiliza el m\'etodo
de DFT (\textit{Density Functional Theory}). La base de este m\'etodo consiste en la introducci\'on de la
\textit{densidad electr\'onica} $\rho$, que viene a representar la probabilidad de encontrar un electr\'on en
el espacio dado un estado de la configuraci\'on del sistema. $\rho$ y $\Psi$ (y por lo tanto $E$) se encuentran relacionadas por

\begin{equation}
    \label{honenberg_kohn}
    \rho(\vec{r}_i) = \int \Psi^{\dagger}(\vec{r}_1, \dots, \vec{r}_n) \Psi(\vec{r}_1, \dots, \vec{r}_n) d\vec{r}_1 \dots d\vec{r}_n \qquad i \in [1,n]
\end{equation}

Donde $\Psi^{\dagger}$ es el conjugado de $\Psi$. La energ\'ia entonces es un \textit{funcional} de la densidad, $E[\rho]$. Otros trabajos de DFT dan m\'as
detalles a esta relaci\'on. Uno de los principales resultados de DFT es la siguiente ecuaci\'on para la relaci\'on entre $E$ y $\rho$:

\begin{equation}
    \label{honenberg_kohn} 
    E[\rho] = T_s[\rho] + V_{ne}[\rho] + \frac{1}{2} \int \int \frac{\rho(\vec{r}_1) \rho(\vec{r}_2)}{r_{12}} d\vec{r}_1 d\vec{r_2} + E_{xc}[\rho]
\end{equation}

Donde $T_s[\rho]$ es la energ\'ia cin\'etica asociada con la densidad, $V_{ne}[\rho]$ es la energ\'ia potencial producto de la interacci\'on entre los
electr\'ones (la densidad) y los n\'ucleos, el tercer t\'ermino es el resultado de la repulsi\'on de Coulomb entre eletrones y $E_{xc}[\rho$ es la
energ\'ia de intercambio y correlacci\'on.

Para este trabajo, debido a su costo computacional~\cite{PaperNitscheManu}, nos interesar\'a sobre todo el computo de la energ\'ia de intercambio y correlacci\'on ($E_{xc}$).
Esta energ\'ia es calculada mediante 

\begin{equation}
    E_{XC} = \int \rho(r) \epsilon_{xc}(\rho(r)) dr
\end{equation}

que es integrada numericamente en una grilla, mediante la suma

\begin{equation}
    E_{XC} \approx \sum_j \rho(r_j) \epsilon_{xc} (\rho(r_j))
\end{equation}

TODO: CONTINUAR ESTO.
