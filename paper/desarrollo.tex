
\subsection{Estructura inicial del programa}

El programa originalmente estaba concebido para ser corrido en GPU nVidia GTX8800.
Luego, las cosas que se usaron para este desarrollo consistieron en la libreria CUDA version
2, para arquitecturas a lo sumo SM20.

La estructura del programa era la siguiente:
\begin{enumerate}
\item Se determinan los mallados del sistema
\item Se clasifican en cubos y esferas
\item Para cada elemento, se resuelve el sistema
\item Esto se repite hasta que converga a lo sumo en una cantidad limitada de pasos, o diverga
\end{enumerate}

La resoluci\'on del sistema en si esta compuesta de varios pasos:
\begin{enumerate}
\item Se obtiene el valor de la funcion de onda en cada punto de la malla
\item Se generan las matrices con los gradientes y los hessianos de cada funcion
\item Se calculan las densidades y el producto matricial por bloques por funcion
\item Se obtienen las derivadas de la densidad calculada antes
\end{enumerate}


\subsection{Cuellos de botellas originales, limitantes estructurales}

El cuello de bottella principal que presentaba este codigo era en la 
resoluci\'on del sistema, especificamente en el codigo que calculaba las densidades. 
En el GPU, esta funci\'on insumia el 95\% del tiempo total, que en sistemas grandes estaba
en el orden de minutos. Los problemas principales que mostraba esta funcion fueron:
\begin{itemize}
\item Cantidad de accesos a memoria global
\item Falta de accesos coalescentes
\item Sobreuso de la memoria compartida
\item Subsaturacion de los SM
\end{itemize}


\subsection{Cambios en el threading}
Cambiamos de blocks por funcion, a blocks por puntos
\subsection{Cambios en las memorias compartidas}
Accesos por bloque
\subsection{Cambios en los accesos globales}
A traves de textura, un acceso menos
\subsection{Cambios en la reduccion}
Hubo que agregar reduccion de suma a nivel punto porque ya no se comparten mas la info
\subsection{Cambios en los pasajes de informacion intrawarp}
Los shuffles que no anduvieron salvo en function
\subsection{Cambios en las operaciones matem\'aticas}
Cambiar los vec\_type4 por 3 en los accesos a la shared es mucho mejor, no hace falta alinear ahi
