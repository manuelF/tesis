La siguiente arquitectura analizada en este trabajo es la arquitectura GPGPU desarrollada por NVIDIA, CUDA.
~\cite{cudaHandbook} CUDA surge naturalmente de la aplicaci\'on de los pipelines desarrollados para
graficos aplicados a computo cientifico.

Desde fines de los 90 que se pronosticaba el fin de la ley de Moore [ref]. Las velocidades obtenidas
entre 1970 y el 2000 en termino de procesadores se dio de acuerdo a dos grandes tecnicas: la miniaturizacion
y el aumento de velocidad de procesador (en Mhz). Alrededor del a\~no 2002, ya no se pudo aumentar
la velocidad de los procesadores sin que el calor generado fuera indisipable. Esto fuerza al modelo
CPU a adoptar multiples cores como la solucion para mantener el aumento de ``performance'', aunque esta
ya no depende de un procesador individual. Cuando esto pas\'o, los desarrolladores de GPU ya tenian
procesadores de multiples nucleos andando hace a\~nos. En los a\~nos sucesivos, algunos investigadores
lograron modelar sus problemas de HPC como aplicaciones de algebra lineal aplicadas a gr\'aficos, que es
lo \'unico que sabian resolver las placas de video. En el 2006, NVIDIA introduce la arquitectura G80,
que es la primera placa de video que deja de resolver \'unicamente problemas especializados a gr\'aficos
para pasar a un motor gen\'erico donde cuenta con un set de instrucciones consistente para todos los
tipos de operaciones que realiza (geometria, vertex y pixelshaders)~\cite{}. Como subproducto de esto,
deja de tener pipelines especializados y pasa a tener procesadores simetricos m\'as sencillos, m\'as
faciles de construir. Esta arquitectura es la que se ha mantenido y mejorado en el tiempo, permitiendo
a las GPU escalar masivamente en procesadores sencillos, de un bajo clock con una disipacion termica
manejable.

Los puntos fuertes de las GPGPU modernas consisten en poder atacar los problemas de paralelismo
de manera pseudo-explicita, y con esto poder escalar ``facilmente'' si solamente se corre en una
placa mas r\'apida. ~\cite{} Tecnicamente esta arquitectura cuenta cientos a miles de procesadores
especializados en cuentas de punto flotante, que trabajan de a bloques de manera sincronica
 que procesan cada uno un \textit{thread}. Cada thread a su vez cuenta con
64 a 256 registros ~\cite{NvidiaFermi} ~\cite{NvidiaKepler}, como porcion de un register file de 64kb.
Las placas cuentas con m\'ultiples niveles de cach\'e y memorias especializadas (subproducto de
su dise\~no fundamental para gr\'aficos). No poseen instrucciones SIMD, ya que su dise\~no primario
esta basado en SIMT (\textit{Single Instruction Multiple Thread}), las cuales se ejecutan en los
bloques sincronicos de procesadores. De este modo, las placas modernas como la K40 alcanzan
poder de computo de 4.3 TFLOPs de precision simple, 1.7 TFLOPs de precision doble y 288GB/sec de
transferencia, usando 2880 CUDA Cores ~\cite{NvidiaKeplerDatasheet}. Para poner en escala, esto haria
que usando dos de estas placas la computadora mas grande del mundo en Noviembre
2001 ~\cite{Top500_2001Noviembre}.

Para poder correr programas explotando la arquitectura CUDA, hay que reescribirlos de manera que
el problema se particione usando el modelo de grilla de bloques de threads. Esto implica una
reescritura completa de los codigos actuales en CPU y un cambio de paradigma importante, al
dejar de tener vectorizaci\'on, paralelizacion automatica y otras t\'ecnicas tradicionales
de CPU. Sin embargo, este trabajo ha rendido sus frutos: en los ultimos 6 a\~nos, la literatura
de HPC con aplicaciones en GPU ha explotado con desarrollos nuevos basados en la aceleracion
de algoritmos num\'ericos (su principal uso).% ~\cite{meter refs a gpu montecarlos}
Adem\'as, no todas las aplicaciones tienen que reescribirse de manera completa, con la introduccion
de las librerias CuBLAS y CuFFTw, se han buscado reemplazar con minimos cambios las historicas
librerias BLAS y FFTw, piedras fundamentales del computo HPC. %~\cite{meter refs paper cublas y fftw}.

Nuevas soluciones para la portabilidad se siguen desarrollando: librerias como Thrust ~\cite{thrust},
OpenMP4.0 ~\cite{spec_omp40}, OpenACC 2.0 ~\cite{spec_oacc2.0} son herramientas que buscan hacer el
c\'odigo agnostico al acelerador de computo que usen. Estas permiten definir las operaciones de
manera gen\'erica y dejan el trabajo pesado al compilador para que parta el problema de la manera
que el acelerador (CPU, GPU, MIC) necesite. Obviamente, los ajustes finos siempre quedan pendiente para
el programador especializado, pero representan estas herramientas un avance fundamental al uso
m\'asivo de tecnicas de paralelizaci\'on autom\'aticas, necesarias hoy dia e imprescindible en el
futuro.
