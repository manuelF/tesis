La siguiente arquitectura analizada en este trabajo es la arquitectura GPGPU desarrollada por NVIDIA, CUDA.
CUDA surge naturalmente de la aplicaci\'on de los pipelines desarrollados para
graficos aplicados a computo cientifico.

Las placas de video, aparecen en 1978 con la introducci\'on de Intel del iSBX 275, permitiendo dibujar lineas,
arcos y bitmaps, comunicada por DMA al procesador principal. En 1985, la Commodore Amiga incluia un coprocesador
gr\'afico que podria ejecutar instrucciones independientemente del CPU. En la decada del 90, multiples
avances surgieron en la aceleraci\'on 2D, para dibujar las interfaces gr\'aficas de los sistemas operativos,
y para mediados de la decada, muchos fabricantes estaban incursionando en las aceleradoras 3D como
add-ons a las placas gr\'aficas tradicionales 2D. A principios de la d\'ecada del 2000, se agregaron los
\textit{shaders} a las placas, que eran peque\~nos programas independientes que corrian nativo en el GPU,
y se podian encadenar entre si, uno por pixel en la pantalla. Este paralelismo es el desarrollo fundamental
que llevaba a las GPU a poder procesar operaciones gr\'aficas ordenes de magnitud m\'as rapidas que el CPU.

En el 2006, NVIDIA introduce la arquitectura G80,
que es la primera placa de video que deja de resolver \'unicamente problemas especializados a gr\'aficos
para pasar a un motor gen\'erico donde cuenta con un set de instrucciones consistente para todos los
tipos de operaciones que realiza (geometria, vertex y pixel shaders) ~\cite{cudaHandbook}. Como subproducto de esto,
deja de tener pipelines especializados y pasa a tener procesadores simetricos m\'as sencillos, m\'as
faciles de construir. Esta arquitectura es la que se ha mantenido y mejorado en el tiempo, permitiendo
a las GPU escalar masivamente en procesadores sencillos, de un bajo clock con una disipaci\'on termica
manejable.

Los puntos fuertes de las GPGPU modernas consisten en poder atacar los problemas de paralelismo
de manera pseudo-explicita, y con esto poder escalar ``facilmente'' si solamente se corre en una
placa mas r\'apida. ~\cite{} Tecnicamente esta arquitectura cuenta cientos a miles de procesadores
especializados en cuentas de punto flotante, que trabajan de a bloques de manera sincronica
 que procesan cada uno un \textit{thread}. Cada thread a su vez cuenta con
64 a 256 registros ~\cite{NvidiaFermi}~\cite{NvidiaKepler}, como porci\'on de un register file de 64kb.
Las placas cuentas con m\'ultiples niveles de cach\'e y memorias especializadas (subproducto de
su dise\~no fundamental para gr\'aficos). No poseen instrucciones SIMD, ya que su dise\~no primario
esta basado en SIMT (\textit{Single Instruction Multiple Thread}), las cuales se ejecutan en los
bloques sincronicos de procesadores. De este modo, las placas modernas como la K40 alcanzan
poder de computo de 4.3 TFLOPs de precision simple, 1.7 TFLOPs de precision doble y 288GB/sec de
transferencia, usando 2880 CUDA Cores ~\cite{NvidiaKeplerDatasheet}. Para poner en escala, esto haria
que usando solo dos de estas placas la supercomputadora m\'as grande del mundo en Noviembre
2001 ~\cite{Top500November2001}.

Para poder correr programas explotando la arquitectura CUDA, se deben escribir de manera que
el problema se particione usando el modelo de grilla de bloques de threads. Esto implica una
reescritura completa de los c\'odigos actuales en CPU y un cambio de paradigma importante, al
dejar de tener vectorizaci\'on, paralelizaci\'on automatica y otras t\'ecnicas tradicionales
de optimizaci\'on en CPU. Sin embargo, este trabajo ha rendido sus frutos en muchos casos:
en los \'ultimos 6 a\~nos, la literatura de HPC con aplicaciones en GPU ha explotado con
desarrollos nuevos basados en la aceleraci\'on de algoritmos num\'ericos (su principal uso).
% ~\cite{meter refs a gpu montecarlos}
Adem\'as, no todas las aplicaciones deben reescribirse de manera completa. Con la introducci\'on
de las librerias CuBLAS y CuFFT, se han buscado reemplazar con minimos cambios las historicas
librerias BLAS y FFTw, piedras fundamentales del computo HPC. ~\cite{cublas} ~\cite{cufft}.

Nuevas soluciones para la portabilidad se siguen desarrollando: las librerias como Thrust ~\cite{thrust},
OpenMP4.0 ~\cite{OpenMPspec} y OpenACC 2.0 ~\cite{OpenACCSpec} son herramientas que buscan hacer el
c\'odigo agnostico al acelerador de computo que usen. Estas permiten definir las operaciones de
manera gen\'erica y dejan el trabajo pesado al compilador para que parta el problema de la manera
que el acelerador (CPU, GPU, MIC) necesite. Obviamente, los ajustes finos siempre quedan pendiente para
el programador especializado, pero estas herramientas representan un avance fundamental al uso
m\'asivo de t\'ecnicas de paralelizaci\'on autom\'aticas, necesarias hoy dia e imprescindible en el
futuro.

La aplicaci\'on LIO ya contaba con una implementacio\'n CUDA desarrollada anteriormente a este
trabajo ~\cite{TesisNitsche}. Ac\'a nos encargaremos de analizar algunos detalles internos de
la arquitectura en esa implementaci\'on, y estudiar el impacto de las distintas mejoras considerando
los cambios que han habido en las iteraciones de CUDA desde entonces.
