\subsection{Arquitectura GPGPU}
La arquitectura de las GPGPU esta enfocada a procesamiento de grandes cantidades de datos
de puntos flotante. El procesador de GPGPU cuenta con cientos de ALU sincronizadas
por bloques, permitiendo un paralelismo adaptativo a distintos problemas.

El procesamiento GPGPU es similar al procesamiento vectorial 
realizado por las supercomputadoras Cray y IBM que surgio en los 1960's.
El procesamiento consiste en en un hibrido entre compilador y procesador. Se determina
un conjunto de elementos a procesar y se elije una manera de dividirlos entre distintos
procesadores mediante distintas keywords del lenguaje. 

El paralelismo principalmente se divide en varios niveles:
-Thread LEvel
-Block Level
-Board level

El paralelismo a nivel de thread permite comunicaciones internas entre las
distintas uniddes de ejecucion.


\subsection{Funcionamiento de un GPGPU}

\subsection{Idoneidad para el problema}
