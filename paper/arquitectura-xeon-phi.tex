\subsubsection{Hardware}

La arquitectura del Xeon Phi, parte de la linea de Intel Many Core Architecture (MIC), est\'a esquematizada
en la figura~\ref{fig:xeon_phi_arch}. Cada procesador esta esquematizado en~\ref{fig:xeon_phi_core}

La base de esta arquitectura consiste entre 60 y 80 cores SMP (\textit{Symmetric Multiprocessing}), con lo cual todos
ellos comparten la misma memoria principal. Cada uno de los cores tiene un \textit{clock rate} de 1 GHz,
con una arquitectura similar al set de instrucciones de Intel IA-32. Las principales diferencias con este
conjunto de instrucciones son el soporte para direccionamiento a 64 bits y nuevas instrucciones de vectorizaci\'on.

Cada procesador tiene adem\'as soporte para 4 \textit{threads}, es decir, 4 hilos de ejecuci\'on diferentes. Adicionalmente,
cada \textit{thread} puede ejecutar dos instrucciones por ciclo de clock, mediante el uso de dos pipes: \textit{V-pipe} y \textit{U-pipe}.
Algunas instrucciones, sin embargo, solo puede ser ejecutadas en una de las dos: por ejemplo las instrucciones de vectorizaci\'on solo pueden
ser ejecutadas en la \textit{U-pipe}. Para ejecutar estas instrucciones se dispone de una unidad de vectorizaci\'on (VPU, \textit{Vector Processing Unit}).
Esta unidad cuenta 32 registros SIMD (Single Instruction Multiple Data) de 512 bits. La latencia de estas instrucciones es de 4 ciclos de clock pero permiten
operar sobre 16 valores de punto flotante de precisi\'on simple (\textit{float}) a la vez.

Adem\'as de la unidad de vectorizaci\'on y la unidad escalar, cada procesador cuenta con 32 Kb de cache L1 y 512 Kb de cache
L2. Estas caches son asociativas \textit{8-way} y su linea de cache tiene 64 bytes. La coherencia de cach\'e se mantiene
mediante un directorio distribuido de \textit{tags} (v\'ease figura~\ref{fig:xeon_phi_arch}) dividido en 64 secciones e implementado
por hardware. La memoria principal consiste de memoria RAM GDDR5 en la placa, de 8 GB con velocidad de transferencia de 5.5 GT/s en 16 canales y con transferencia de 4
bytes. Esto nos da un l\'imite de ancho de banda te\'orico de 352 GB/s pero detalles de implementaci\'on de los chips limitan este valor a 200 GB/s.

Cada procesador, puede realizar pedidos de memoria independientes sin que la memoria se convierta en un cuello
de botella importante~\cite{Fang}. Sin embargo, los 4 \textit{threads} dentro de un \textit{core} ven sus accesos a memoria serializados.

Adicionalmente los cores est\'an conectados por dos anillos bidireccionales que les permite comunicarse entre si. La velocidad de
comunicaci\'on es suficiente para considerar que todos los procesadores son sim\'etricos (es decir, cada procesador puede comunicarse con
cualquier otro con un \textit{overhead} despreciable)~\cite{Fang}.

Por \'ultimo, cada procesador tiene un \textit{in-order pipeline} de corta longitud, diferencia importante con los cores de un procesador
estandar de la arquitectura x86. El \textit{pipeline} corto implica que las operaciones escalares no tienen latencia y las vectoriales tienen baja latencia,
y el costo por \textit{branch misprediction} es bajo~\cite{IntelXeonWhitePaper}. Este punto diferencia fuertemente al Xeon Phi de aceleradores de computo como las GPGPU, 
que tienen alto costo en las bifuraciones de decisiones. Sin embargo, el Xeon Phi no ejecuta instrucciones de manera \textit{out-of-order}, lo cual implica que muchas 
t\'ecnicas de optimizaci\'on que explotan el paralelismo a nivel instrucci\'on usuales en arquitecturas como x86-64 no son aplicables.

La otra diferencia es nuevas instrucciones de vectorizaci\'on, incompatibles con sets de vectorizaci\'on anteriores de Intel (por ejemplo AVX o SSE 4.1).
Estas operaciones incluyen implementaciones por \textit{hardware} de operaciones com\'unes en HPC: rec\'iproco de un valor, ra\'iz cuadrada, potencia y
exponenciaci\'on, y operaciones m\'as relacionadas con la memoria como por ejemplo \textit{scatter and gatter} y stores \textit{streameados} de manera de aprovechar
mejor el ancho de banda que tiene la arquitectura.

El Xeon Phi es un coprocesador, lo cual implica que necesita ser instalado sobre una computadora que sirva de \textit{host}. La comunicaci\'on con este host
ocurre a trav\'es de un bus PCI Express, no comparten ni memoria ni otros perif\'ericos como por ejemplo disco duro. Existen dos m\'etodos para acceder al Xeon~\cite{BookXeonPhi}

\begin{enumerate}
    \item Nativo: El Xeon Phi permite correr c\'odigo directamente, mediante el uso de SSH (Secure SHell). Esto es gracias a la presencia de BusyBox Linux como sistema operativo,
    lo cual da soporte de sistema de archivos y entorno de ejecuci\'on. La interfaz utiliza TCP/IP virtualizado mediante el bus PCI Express. Si bien el coprocesador no tiene acceso a \textit{storage} persistente (puesto que el sistema operativo esta montado sobre la memoria) esto puede resolverse utilizando un sistema de archivos remoto montado en la memoria del host.
    \item Offloading: El \textit{host} puede delegar la ejecuci\'on de ciertas porciones de c\'odigo al coprocesador. Esto requiere que los datos necesarios para el c\'omputo sean copiados del \textit{host} al Xeon Phi, lo cual puede implicar que el bus puede ser un cuello de botella importante (puesto que los datos de entrada, y la salida deben ser movido al Xeon Phi y traidos de vuelta al finalizar el c\'omputo).
    \item Simétrico: En este modo de ejecuci\'on se piensa al Xeon Phi y su host como dos nodos en un \textit{cluster} de c\'omputo, y al bus PCIe como una red de alta velocidad.
Este modo es especialmente interesante si se dispone de m\'as de un Xeon Phi en un mismo host, y se utiliza una interfaz de pasado de mensajes entre ellos como por ejemplo MPI (\textit{Message Passing Interface}).
\end{enumerate}

Por \'ultimo, en pos de simplificar el trabajo de adaptar una aplicaci\'on a usar el Xeon Phi, el mismo provee una unidad de monitoreo de \textit{performance} (\textit{Performance
Monitoring Tool}, PMU). Esta unidad permite la colecci\'on de informaci\'on del coprocesador, aunque no tiene soporte para ciertas caracter\'isticas com\'unes a los procesadores
de lineas m\'as est\'andar de Intel (por ejemplo: \textit{sampling} preciso de eventos por \textit{hardware}, como por ejemplo ciclos por instrucci\'on o \textit{misses} de cache).
