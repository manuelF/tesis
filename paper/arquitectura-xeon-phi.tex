La arquitectura del Xeon Phi, parte de la linea de Intel Many Core Architecture (MIC), esta esquematizada
en la figura~\ref{fig:xeon_phi_arch}. Cada procesador esta esquematizado en~\ref{fig:xeon_phi_core}

La base de esta arquitectura consiste en 61 procesadores SMP (Symmetric Multiprocessing), con lo cual todos
ellos comparten la misma memoria. Cada uno de estos procesadores tiene un \textit{clock rate} de 1009 MHz, 
con una arquitectura basada en el set de instrucciones de Intel IA x86. Las principales diferencias con este
set de instrucciones son el soporte para direccionamiento a 64 bits y nuevas instrucciones de vectorización. 

Cada procesador tiene además soporte para 4 threads por hardware. Adicionalmente, cada core tiene una unidad
de vectorización (VPU, \textit{Vector processing unit}). Esta unidad tiene 32 registros SIMD (Single Instruction
Multiple Data) de 512 bits. Para utilizar estos registros se dispone del set de instrucciones especial de vectorización, 
que es imcompatible con sets de instrucciones anteriores como por ejemplo SSE (Streaming SIMD Extensions) y AVX (Advanced
Vector Extensions) de x86-64.

Además de la unidad de vectorización y la unidad escalar, cada procesador cuenta con 32 Kb de cache L1 y 512 Kb de cache
L2. Estas caches son asociativas \textit{8-way} y su linea de cache tiene 64 bytes. La coherencia de caché se mantiene
mediante un directorio distribuido de \textit{tags} (véase figura~\ref{fig:xeon_phi_arch}) dividido en 64 pedazos e implementado
por hardware. La memoria principal consiste de memoria RAM GDDR5, de 8 GB con velocidad de transferencia de 5.5 GT/s. 
Cada procesador, de acuerdo a [Fang], puede realizar pedidos de memoria separados sin que la memoria se convierta en un cuello
de botella importante. Sin embargo, los 4 \textit{threads} dentro de un \textit{core} ven sus accesos a memoria serializados.

Adicionalmente los procesadores están conectados por dos anillos bidireccionales que les permite comunicarse. La velocidad de
comunicación es suficiente, de acuerdo a los microbenchmarks realizados en [Fang], para considerar que todos los procesadores son
simétricos entre si. 

Por último, cada procesador tiene un \textit{in-order pipeline} de corta longitud, diferencia importante con los cores de un procesador
estandar de Intel. El \textit{pipeline} corto implica que las operaciones escalares no tienen latencia y las vectoriales tienen baja latencia,
y el costo por \textit{branch misprediction} es bajo. Este punto diferencia fuertemente al Xeon Phi de aceleradores de computo como las GPU
(Graphic Processing Unit), que tienen alto costo en las bifuraciones de decisiones. Sin embargo, el Xeon Phi no ejecuta instrucciones de manera
\textit{out-of-order}, lo cual implica que muchas operaciones de paralelismo a nivel instrucción usuales en arquitecturas como IA32 o x86-64 no son
aplicables.

El Xeon Phi es un coprocesador, lo cual implica que necesita ser instalado en un \textit{host}. La comunicación con este host
ocurre a través de un bus PCI Express. Existen dos métodos para acceder al Xeon.

\begin{enumerate}
    \item Nativo: El Xeon Phi permite correr código directamente, mediante el uso de SSH (Secure SHell). Esto es gracias al
    uso de BusyBox Linux como sistema operativo, lo cual da soporte de sistema de archivos y entorno de ejecución. Si bien no
    tiene acceso a \textit{storage} permanente esto puede resolverse gracias al \textit{stack} TCP/IP disponible en BusyBox.
    \item Offloading: El \textit{host} puede delegar la ejecución de ciertas porciones de código al coprocesador. Esto requiere
    que los datos necesarios para el cómputo sean copiados del \textit{host} al Xeon Phi, lo cual puede implicar que el bus puede
    ser un cuello de botella importante.
\end{enumerate}
