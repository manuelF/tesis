\subsection{Objetivos}

El principal objetivo de este trabajo es analizar distinas opciones dentro de
las ofertas de plataformas para HPC (\textit{High Performance Computing}) con
el propósito de presentar puntos de \textit{trade-off} entre las mismas. Con la
intención de realizar esta comparación de una manera lo más realista posible,
se utilizará como caso de estudio un software de dinámica molecular (LIO) que
ha sido probado en diversas ocasiones y sobre plataformas de \textit{hardware}
muy heterogéneas. Asimismo, nuestra comparación buscará ser justa en términos
de la \textit{performance} obtenida para cada una de las arquitecturas
examinadas, pero sin dejar de tener en cuenta factores tan importantes a la
hora de afrontar un desarrollo de magnitud tal como lo es portar una
aplicación ya existente: la cantidad de código que debe ser reescrito, la
mantenibilidad del código resultante y la disponibilidad de recursos para
asistir en el proceso.

\subsection{Problema abordado}

\subsection{Arquitecturas estudiadas}

\subsubsection{NVIDIA CUDA}

\subsubsection{Intel IA 64}

\subsubsection{Intel Xeon Phi}

La última de las arquitecturas examinadas en este trabajo es un desarrollo relativamente reciente
por parte de Intel, el coprocesador Xeon Phi. Desde su introduccion al mercado en el a\~no 2012, se ha visto implementaciones de gran magnitud con esta tecnología, como puede verse en el papel
que juega dentro de supercomputadoras como la Tianhe-2 de la Universidad de Sun Yat-Sen en China,
listada en TOP 500 como la supercomputadora más rápida del mundo en Junio 2013, Noviembre 2013 y
Junio 2014. Esta supercomputadora de 16000 nodos, cada uno de los cuales consiste de dos Intel
Ivy Bridge Xeon en combinación con 3 coprocesadores Xeon Phi, es teóricamente capaz de alcanzar los
54.9 petaflops de computo, y superó por casi el doble de flops a su competidor más cercano en
Junio de 2013.

Los principales puntos fuertes de esta nueva arquitectura no son solo sus especificaciones tecnicas, que prometen una
capacidad de computo de 1 TFLOPs y 240 GB/s de transferencia de memoria, sino que además ha de considerarse
que, a diferencia de sus competidores directos NVIDIA CUDA y OpenCL, el esfuerzo de portar la aplicación a utilizar
el Xeon Phi es presentado como mucho menor por parte de Intel. Esta combinación de performance y portabilidad
presenta al Xeon Phi como una alternativa muy interesante en el campo de aplicaciones intensivas en cómputo.

Para lograr esto, Intel desarroll\'o una nueva arquitectura de muchos procesadores (Many Integrated Core Architecture, MIC) de
tipo SMP (Symmetric MultiProcessing), bajo el nombre en código \textit{Knights Corner} y similar a desarrollos anteriores como \textit{Larrabee}. El coprocesador dispone de 60 cores basados
en el dise\~no del Intel Pentium, con una ISA (Instruction Set Architecture) muy similar al set de instrucciones IA-32, mas no compatible con el mismo. Además de su cantidad
numérica, estos cores disponen de registros SIMD (Single Instruction Multiple Data) de 512 bits (el doble de tamaño que los registros de 256
bits que disponen los procesadores de Intel existentes en el set de instrucciones AVX de x86). Esto implica que programas
que ya explotan fuertemente el paralelismo a nivel datos y a nivel \textit{threads} de computo debieran poder hacer uso de
estas nuevas prestaciones con una recompilación del c\'odigo de la aplicaci\'on. Adicionalmente el coprocesador puede usarse para que el
\textit{host} delege calculos intensivos en el mismo (con el cual esta conectado por un bus
PCI Express de alta velocidad) y luego obtenga los resultados del mismo. Este patr\'on de uso es similar a los aceleradores de computo como las GPGPU modernas. Ambos modos son soportados por el \textit{toolchain} que provee
Intel para desarrollo de software, que consiste no solo de compiladores de C, C++ y Fortran, sino que también de librerías especialmente optimizadas para uso del Xeon Phi
como lo es la Intel MKL (Math Kernel Library).

Sin embargo, factores diversos hacen que si bien el esfuerzo de portar una aplicación ya existente para aprovechar el Xeon Phi es menor en principio
a otras opciones como NVIDIA CUDA, el mismo no es insignificante. La cantidad de cores se ve compensada por su relativa baja velocidad de clock
(aproximadamente 1.0 GHz), lo cual requiere mas esfuerzo por parte del programador para hacer buen uso de la escalabilidad tanto en cores como en
paralelismo en uso de los datos (vectorización), entre otros aspectos.

Los detalles particulares de esta arquitectura y su impacto serán descriptos y analizados en capítulos posteriores.
La literatura con respecto a esta plataforma en Q3 de 2014 es existente, aunque aun escasa~\cite{Fang}. Estos trabajos
exploratorios buscan analizar distintos aspectos del uso del coprocesador a trav\'es de  modelos sencillos (microbenchmarks, optimizar aplicaciones de peque\~no tama\~no). En este trabajo buscamos
poner las recetas y consejos obtenidos para el uso del Xeon Phi en contexto de tecnolog\'ias competidoras ya establecidas, dentro del marco de portar una aplicaci\'on cient\'ifica de mediano tama\~no ya en uso.

