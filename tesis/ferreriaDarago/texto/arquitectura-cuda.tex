\subsection{CUDA}

\subsubsection{Organizacion de procesadores}

El c\'odigo escrito para CUDA se puede compilar a dos targets distintos; uno es
el codigo binario nativo de la placa target donde se va a correr y el otro es un
codigo intermedio, llamado c\'odigo PTX, que se JIT compila por el driver CUDA
antes de enviar a la placa, de modo que sea portable entre placas y arquitecturas
(retrocompatibles).

Cuando se lanza un kernel de ejecuci\'on, este c\'odigo binario se carga en la DRAM propia
de la placa. Este codigo va a ser ejecutado por todos los bloques que se hayan definido cuando
se lanz\'o el kernel. El ``Thread Engine'' del GPGPU se encarga de repartir los bloques a los
SM (Streaming Multiprocessors). Cada SM luego va a ejecutar de a grupos de 32 threads, cada uno
de ellos corriendo sobre un SP (Streaming Processor). A traves de sus 2 o 4 unidades de dispatch de warp
(de acuerdo a la generaci\'on de los chips), el SM puede dinamicamente cambiar el warp que se esta ejecutando en sus SP,
escondiendo la latencia de los stalls de pipelines forzosos para la ejecucion de instrucciones de multiples clocks.
Estos threads a su vez obtienen sus registros de un register file com\'un a todos los SP. Un scoreboard
es mantenido por cada dispatcher de warps para poder determinar que threads estan listos para correr. En Kepler
este scoreboard es simplificado ya que las latencias de las operaciones matem\'aticas es conocido, por
lo cual se puede reemplazar por contadores m\'as sencillos. ~\cite{NvidiaKepler}

Todos los SP de un GPGPU son sim\'etricos y cuentan todos con las mismas unidades de FP y de Enteros,
y entre varios cores comparten una o dos unidades de precisi\'on doble y una unidad de funciones especiales.
En Fermi no se pueden ejecutar simultaneamente instrucciones de precision doble y simple, y como
las de doble requerien ambos pipelines, suelen disminuir considerablemente el aprovechamiento
de los cores. En Kepler, al usar 4 unidades de despacho de warp, puede elegir una mejor
combinaci\'on para poder ejecutar simultaneamente instrucciones que no dependan de las mismas
unidades funcionales en el pipeline. ~\cite{NvidiaKepler}

Como estos procesadores implementan el estandar IEEE754-2008, cuentan con precision simple y
doble correcta, por lo cual dentro de las operaciones nativas, cuentan con instrucciones FMA
(Fused multiply-add) que no pierden precisci\'on en etapas intermedias de redondeo.

%Como el c\'odigo se ejecuta de manera sincr\'onica entre todos los threads del warp, las instrucciones
%condicionales proveen un problema para esta arquitectura. Como las distintas ramas del condicional
%son instrucciones excluyentes, no las deben ejecutar todos los threads simultaneamentes. El procesador
%deshabilitaba los cores que manejaban los threads de las ramas que no se ejecutaban del
%condicional~\cite{NvidiaTesla}. Para disminuir la cantidad de instrucciones de condicionales
%que se ejecutan, se cuenta con ~\cite{NvidiaFermi} predicaci\'on en todas las instrucciones de la ISA.


\subsubsection{Niveles de memoria}

La arquitectura CUDA esta enfocada a procesamiento de grandes cantidades de datos
de puntos flotante. El procesador GPGPU cuenta con cientos de ALU sincronizadas
por bloques, permitiendo un paralelismo adaptativo a distintos problemas.

El procesamiento GPGPU es similar al procesamiento vectorial
realizado por las supercomputadoras Cray y IBM que surgio en los 1960's, pero
en vez de usar VLIW para procesamiento masivo, CUDA usa multiples hilos de ejecuci\'on
que trabajan simultaneamente sobre los datos.
El procesamiento consiste en un funcionamiento hibrido entre compilador y procesador. Se determina
un conjunto de elementos a procesar y se elije de manera explicita una manera de particionar el
problema a la hora de ser enviado para procesado a la placa.


Para realizar el computo, esta arquitectura cuenta a su vez con multiples clases de memorias
que se adaptan de maneras diferentes a los distintos procesos. Estas incluyen:

\begin{itemize}
  \item Memoria global
  \item Registros
  \item Memoria local
  \item Memoria compartida
\end{itemize}

La memoria global es la memoria principal fuera del chip del GPGPU. Esta es de gran tama\~no
y es compartida por todos los procesadores de la GPGPU y los CPU que integran el
sistema. Es decir, tanto los GPGPU y los CPU pueden invocar las funciones del driver CUDA
para poder transferir datos entre la placa y la memoria RAM. Esta memoria es extremadamente
lenta en comparaci\'on con el procesador, por lo cual se debe siempre intentar recurrir
a las demas memoria que componen la jerarquia del GPGPU antes que acceder seguido a la memoria
global.

Los registros son la unidad basica de almacenamiento de los kernels de ejecuci\'on.
Cada thread de ejecuci\'on cuenta con una cantidad limitada de registros de punto flotante de
32 bits con latencia de un par de ciclos de clock. A su vez, existen una cantidad finita de
registros totales que cuenta un chip (oscilan entre 16.535 y 65.535 registros).

La memoria compartida, o \textit{shared} es una memoria que es visible para todos los threads dentro
de un bloque, pero es independiente entre bloques. Cada thread puede escribir en cualquier
parte de la memoria compartida de su bloque y puede ser leido por cualquier otro thread
de su bloque. Es una memoria muy r\'apida, que tarda una peque\~na cantidad de ciclos
de acceso de lectura. Esta memoria es compartida con la cach\'e nivel 1. Es limitada en
tama\~no y cuenta con una capacidad de entre 32Kb y 64kb que debe ser compartida por todos los bloques.

La memoria local es una memoria propia de cada thread, que se encuentra localizada dentro de la
memoria global. Esta memoria es usada automaticamente por el compilador y sirve como area de spilling
de registros cuando se acaban. Cuenta con las mismas desventajas que la memoria global, incluyendo
su tiempo de acceso.

Adicionalmente, la GPGPU cuenta con multiples niveles de memorias cach\'e para poder aminorar el hecho
de que el principal cuello de botella del computo es la latencia en los accesos a memoria global.
Estas se dividen en tres:

\begin{itemize}
  \item Cach\'e L1
  \item Cach\'e L2
  \item Cach\'e de textura
\end{itemize}

La cach\'e L1 es dedicada por SM. Esta cach\'e fue introducida en Fermi y su dise\~no hace que
tambien esta dedicada a la memoria compartida, por lo que es posible en tiempo de ejecuci\'on
darle directivas a la GPGPU que asigne m\'as memoria cach\'e o m\'as memoria compartida,
permitiendo a los bloques tener mayores espacios de memorias compartidas o mayores hit rate de cach\'es.

La cach\'e L2 es com\'un a todos los SM de la GPGPU, donde, a partir de Fermi en NVIDIA, todos
los accesos de lectura y escritura a memoria global y textura pasan a traves de esta. ~\cite{NvidiaFermi}

La cach\'e de textura es una cach\'e que presenta no solo localidad espacial, como la mayoria
de las cach\'es de procesadores normales (i.e. dato-1, dato, dato+1, etc.), sino que se le
puede agregar el concepto de dimensiones, para poder modelar datos en mas de una dimension.
Esto la convierte en una herramienta clave a la hora de minimizar los accesos a matrices
no solo por filas sino por columnas. Esta memoria se debe definir en momento de compilacion,
en el codigo y no es automatica, porque esta cach\'e tiene limites espaciales (necesarios
para poder definir areas de memoria sobre la cual operar) y a su vez se debe acceder
a los dayos subyacentes a traves de funciones especificas. Una caracteristica adicional
de esta cach\'e es que como necesita resolver estos accesos extra\~nos a la memoria, cuenta
con una unidad propia de resoluci\'on. Esta unidad tiene limitantes a cuanto podemos
exigirle, ya que no posee un ancho de banda suficiente como para resolver todos los
accesos a memoria globales podria servir, asi que hay que usarla judiciosamente.

Para soportar una arquitectura masivamente paralela, se debe usar una ISA
(\texttt{Instruction Set Architecture}) dise\~nada especialmente para el problema. Esta ISA
debe poder soportar conceptos fundamentales del computo GPGPU: grandes cantidades de registros,
operaciones en punto flotante de precision simple y doble y fused multiply-add. Adem\'as,
el c\'odigo compilado para GPGPU debe ser agn\'ostico al dispositivo que lo va a correr, por
lo cual la paralelizaci\'on no debe estar demasiado atada a este, sino que el dispatching
lo debe poder determinar el driver de la placa en tiempo de ejecuci\'on. De este modo, la paralelizaci\'on
deberia ser la m\'as apropiada en cada dispositivo, sin importar la generaci\'on que sea. Un \'ultimo
requerimiento clave de esta ISA es que debe poder soportar poder realizar ajustes manuales,
para poder construir partes claves de ciertas librerias frecuentemente usadas (como las BLAS)~\cite{NvidiaFermi}.


\subsubsection{Niveles de paralelismo}
El paralelismo en CUDA principalmente se divide en varios niveles:
\begin{itemize}
  \item Block Level
  \item Grid Level
  \item Board level
\end{itemize}

El paralelismo a nivel de block agrupa de a bloques 32 threads (llamados \texttt{warp} en CUDA).
Cada uno de estos threads va a contar con el mismo blockId y su propio threadId, identificandolos univocamente.
Estos threads van a correr simultaneamente en el mismo SM (\texttt{Streaming Multiprocessor}) y
van a ser puestos y sacados de ejecuci\'on dinamicamente por el scheduler de hardware que cuenta la GPU.

El paralelismo a nivel de grid determina un arreglo de bloques de ejecuci\'on que particiona
el dominio del problema. El block scheduler va a ejecutar los distintos bloques a ejecutar hasta
el final de la ejecuci\'on de todos los threads. Cuando este bloque termine, y si se cuentan con
los recursos apropiados, se pondr\'a a correr el siguiente bloque disponible.

El paralelismo a nivel de placa consiste en poder distribuir la carga del problema entre distintas
GPGPU dispuestas en un mismo sistema compartiendo una memoria RAM com\'un como si fuera un software
multithreaded tradicional. CUDA no cuenta con un modelo implicito de paralelismo entre distintas placas,
pero es posible hacerlo manualmente eligiendo de manera explicita que dispositivo usar.

\subsubsection{Diferencias entre Tesla, Fermi, Kepler}

\subsubsection{Herramientas de desarrollo, profiling, exploracion}

\subsubsection{Requerimientos de un problema para GPGPU}
En un principio, al ser GPGPU una arquitectura con completo poder expresivo, se puede
calcular cualquier cosa. Sin embargo, hay altos costos en los que se incurren antes de
poder ejecutar c\'odigo. Un problema debe exhibir al menos las siguientes caracteristicas
para que valga la pena poder pensar en correrlo para GPGPU.
\begin{enumerate}
  \item \label{req:paralelo} El problema debe tener una gran parte paralelizable.
  \item \label{req:float} El problema debe consistir mayormente de operaciones numericas de punto flotante.
  \item \label{req:matrix} El problema debe poder ser modelado mayormente en arreglos o matrices.
  \item \label{req:transf} El tiempo de computo debe ser muy superior al tiempo de transferencia de datos.
\end{enumerate}

Item \ref{req:paralelo} se refiere a que debe existir alguna forma de partir el problema
en subproblemas que puedan realizarse simultaneamente, sin que haya dependencias de
resultados entre si. Si el problema requere partes seriales, lo ideal es que se las
pueda concebir las partes paralelas sean etapas de un pipeline de procesos, donde
cada una de estas exhiban caracteristicas fuertemente paralelas. Como las arquitecturas
masivamente paralelas tienen como desventaja una menor eficiencia por core, si el
problema no se puede dividir para maximizar la ocupancion de todos los cores disponibles,
va a ser muy dificil superar en eficiencia a los procesadores seriales.

Item \ref{req:float} habla de que el m\'etodo de resoluci\'on de los problemas debe
consistir del uso de m\'etodos numericos. El set de instrucciones de las arquitecturas
de GPGPU estan fuertemente influenciados por las aplicaciones 3D que las impulsaron
en un principio. Estas consisten mayormente de transformaciones de \'algebra lineal
para modelar luces, hacer renders o mover puntos de vistas. Todos estos problemas
son inherentemente de punto flotante, por lo cual el set de instrucciones, las ALUs
internas y los registros estan optimizados para este caso de uso. Las operaciones
en numeros enteros no son el fuerte de esta arquitectura y suelen ser realizados
m\'as eficientemente por procesadores de proposito general.

Item \ref{req:matrix} menciona que los problemas que mejor se pueden tratar en esta
arquitectura se pueden representar como operaciones entre arreglos o matrices de
dos, tres o cuatro dimensiones. Las estructuras de datos que no estan secuenciales
en memoria pueden incurrir en multiples accesos a esta para recorrerlas, y en las
arquitecturas GPGPU son estos los que generan el mayor cuello de botella. Ademas,
suelen ser dificiles de paralelizar en m\'ultiples subproblemas. Tener como parametros de
entrada matrices o arreglos que se puedan partir facilmente incurre en minimos
overheads de computo y permite aprovechar mejor las memorias caches y las herramientas de
prefetching que brinda la arquitectura.

Item \ref{req:transf} ataca uno de los puntos criticos de esta arquitectura. Para poder
operar con datos, se requiere que estos esten en la memoria de la placa, no la memoria
de proposito general de la computadora. Esto quiere decir, que se deben hacer copias
explicitas entre las memorias, ya que ambas tienen espacios de direcciones independentientes.
Esta copia se realiza a traves de buses que, a pesar de tener un enorme throughput de
datos, tambien tienen una gran latencia (orden de milisegundos). Por lo tanto, para minimizar
el tiempo de ejecuci\'on de un programa usando las GPGPU, se debe considerar tambien el
tiempo de transferencia de datos a la hora de determinar si el beneficio de computar en
menor tiempo lo justifica. Las nuevas versiones de CUDA buscan brindar nuevas herramientas
para simplificar este requerimiento, proveyendo espacio de direccionamiento unico \ref{CUDA4} y
memoria unificada \ref{CUDA6}, pero siguen siendo copias de memoria a traves de los
buses (aunque asincronicas).

Estas caracteristicas limitan enormemente la clase de problemas que una GPGPU puede
afrontar, y suelen ser una buena heuristica para determinar de antemano si vale la pena
invertir el tiempo necesario de la implementaci\'ony ajuste fino.


\subsubsection{Diferencia entre CPU y GPU - Procesadores especulativos}
Viendo la historia de CPU, podemos hacer un recorrido de los problemas que los dise\~nadores de procesadores
fueron enfrentandose a lo largo del tiempo mirando los componentes que fueron apareciendo.

(insertar foto de die de nehalem)

Caches - Pipelines - Predictores - M\'as cach\'e - M\'as controladores de memoria - Multiple issues

Lo importante ac\'a es observar el patron: ``no desechar algo que podramos necesitar pronto'',
``intentar predecir el futuro de los condicionales'', ``intentar correr multiples instrucciones a la vez
porque puede llegar a bloquear en alguna de ellas.''

Todos estos problemas convierten al CPU en una maquina que gira alrededor de la especulacion,
de los valores futuros que van a tener las ejecuciones, del probable reuso de datos.
Podemos observar entonces que las unidades que verdaderamente hacen la ejecuci\'on de las operaciones,
las ALU, son muy pocas en comparaci\'oncon la cantidad de dispositivos de soporte que estan
sobre el die del CPU.

En contraste, las arquitecturas GPGPU son verdaderas maquinas de computo masivo. Estan dise\~nadas para
resolver una y otra vez operaciones muy bien definidas (intrucciones de punto flotante en su mayoria).
Comparativamente con un CPU, las ALU de las GPU son bastante pobres y lentas. No funcionan a las mismas
velocidades de clock y deben estar sincronizadas entre si. Pero la gran ventaja esta en la cantidad.
Un CPU cuenta con unas pocas ALU por core, dependiendo de su algoritmo de scheduling interno
(alrededor de 16 cores por die de x86 es el tope de linea ofrecido actualmente). Un GPU cuenta con miles de ALUs en total
(m\'as de 2500 CUDA Cores en Tesla K20). Un GPU, en cambio, dispone de pocas unidades de soporte del procesamiento.
No cuenta con pipelines especulativos, el tama\~no de las caches
estan a ordenes de magnitud de las de CPU, la latencia a las memorias principales de la GPU estan a
decenas de clocks de distancia, etc. La arquitectura asume que siempre va a tener m\'as trabajo
para hacer, por lo cual en vez de intentar solucionar los pitfalls de un grupo de threads, directamente
los reschedulea para m\'as adelante y continua procesando otro warp de threads. Se puede notar que durante del
dise\~no de la arquitectura GPGPU buscaron resolver el problema del computo masivo pensando en hacer
m\'as cuentas a la vez y recalcular datos viejos de ser necesario. Esto es una marcada diferencia contra
los CPU, que estan pensado en rehacer el menor trabajo posible y intentar mantener todos los datos que pueda en
las memorias caches gigantescas.

Como los CPU tienen que poder soportar cualquier aplicacion, no pueden avocarse de lleno a una sola
problematica. El hecho de no tener que dise\~nar un procesador de proposito general permitio un cambio radical
a la hora de pensar como concebir una arquitectura de gran throughput auxiliar al procesador, no reemplazandolo
sino mas bien adicionando poder de computo.~\cite{GlaskowskyFermi}

Las arquitecturas Fermi (y agregar AMD) han concebido el dise\~no de un procesador de alto desempe\~no.
Su meta principal es poder soportar grandes cantidades de paralelismo, mediante el uso de procesadores
simetricos, pero tomando la fuerte restricci\'on de ``no siempre tiene que andar bien''. Es decir, ellos
mismos asumen que el codigo que van a correr esta bien adaptado a la arquitectura y no disponen
casi de mecanismos para dar optimizaciones post-compilacion. Relajar esta restricci\'on
les permitio romper con el modelo de computo de CPU y definir nuevas estrategias de paralelismo,
que no siempre se adaptan bien a todos los problemas, pero para el subconjunto de los desafios que se
presentan en el area de HPC y de videojuegos han probado ser un cambio paradigm\'atico.

\subsubsection{Idoneidad para el problema}
El problema de QM/MM enfrentado en este trabajo cuenta con mutiples operaciones matematicas de gran
volumen de calculos. En particular, las operaciones matriciales son los cuellos de botella en este
problema.
Para obtener los valores numericos de densidad buscados en los puntos, se deben obtener las derivadas primeras
y segundas, lo cual implica hacer multiples operaciones de multiplicaci\'onmatricial. Estas multiplicaciones
pueden ser paralelizadas por columnas y por filas. Estos problemas estan fuertemente estudiados como paralelizarlos
en la literatura, ya que los problemas de algebra lineal los requieren constantemente.
En nuestro caso, se requieren para un sistema cientas de estas multiplicaciones, algunas con matrices de mas de
$500^2$ elementos. Al consistir este proyecto un sistema de resoluci\'on num\'erico de las f\'ormulas de QM/MM,
los problemas enfrentados eran casi en su totalidad de operaciones de punto flotante. Luego, dados las
caracteristicas de contar con un fuerte nivel de paralelismo en los cuellos de botella y de ser operaciones
mayormente de punto flotante, determinamos que el uso de las GPGPU era idoneo, en comparaci\'on con arquitecturas
de proposito general con menos GFLOPS.

