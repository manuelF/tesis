\subsection{Estructura inicial del programa}

El programa originalmente estaba concebido para ser corrido en GPU nVidia GTX8800.
Luego, las cosas que se usaron para este desarrollo consistieron en la libreria CUDA version
2, para arquitecturas a lo sumo SM20.

La estructura del programa era la siguiente:
\begin{enumerate}
\item Se determinan los mallados del sistema
\item Se clasifican en cubos y esferas
\item Para cada elemento, se resuelve el sistema
\item Esto se repite hasta que converga a lo sumo en una cantidad limitada de pasos, o diverga
\end{enumerate}

La resoluci\'on del sistema en si esta compuesta de varios pasos:
\begin{enumerate}
\item Se obtiene el valor de la funci\'on de onda en cada punto de la malla
\item Se generan las matrices con los gradientes y los hessianos de cada funci\'on
\item Se calculan las densidades y el producto matricial por bloques por funci\'on
\item Se calcula la matriz de coeficientes de las fuerzas entre las particulas
\end{enumerate}


\subsection{Cuellos de botellas originales, limitantes estructurales}

El cuello de botella principal que presentaba este c\'odigo era en la
resoluci\'on del sistema, especificamente en el c\'odigo que calculaba la matriz de Kohn-Sham.
En el GPU, esta funci\'on insumia el 95\% del tiempo total, que en sistemas de gran cantidad
de puntos, estaba en el orden de minutos. Los problemas principales que mostraba esta funci\'on fueron:
\begin{itemize}
\item Cantidad de accesos a memoria global
\item Falta de accesos coalescentes
\item Sobreuso de la memoria compartida
\item Subsaturacion de los SM
\end{itemize}

\subsubsection{Accesos a memoria global}

\subsubsection{Falta de accesos coalescentes}
Asi como a los CPU les interesa realizar accesos alineados a memoria, a los GPU les
interesa aun mas. El termino coalescencia de memoria se aplica a GPU mediante
la organizacion de los accesos a memoria de manera ordenada y predecible. La l\'ogica
detras de esto es que cuando el GPU accede a memoria de manera alineada, puede traer
entre 16 y 64 bytes en una sola lectura, mientras que si debe acceder de manera no
alineada a memoria, o con threads que no acceden de manera predecible a esta, entonces
se deberan serializar los accesos y separar en m\'ultiples accesos a memoria global para leer
esta informaci\'on. Este problema solamente se agrava si se debe hacer muy frecuentemente
por cientos de threads, como es el caso de los bloques con gran nivel de paralelismo explicito.

\subsubsection{Sobreuso de la memoria compartida}
Los SM cuentan con una memoria compartida entre todos los threads, la memoria shared. Esta
tiene un bus de baja latencia con conexionado especifico dentro de cada SM.


\subsubsection{Subsaturacion de los SM}
La subsaturacion de los SM se da en los casos donde haya SM que esten listos para correr
c\'odigo pero que no puedan hacerlo porque tienen contencion en algunos de sus recursos.
Este problema es dificil de rastrear, porque hay principalmentre tres recursos que,
en un principio, parecen ilimitados pero en realidad son finitos y compartidos por
los procesadores de la GPGPU. Estos son:
\begin{itemize}
\item Cantidad total de threads por bloque.
\item Cantidad total de registros usados por thread.
\item Cantidad de memoria compartida por bloque.
\end{itemize}

El mecanismo de scheduling de los SM funciona asignando un bloque a cada SM, que
va a correr hasta que terminen todos sus threads asignados. Idealmente, cada
bloque cuenta con una cantidad de threads suficiente para poder esconder la latencia
de las ejecuciones mediante un cambio de contexto. La arquitectura GPGPU esta dise\~nada
para este fin, por lo cual se cuenta con un cambio de contexto de costo cero para
poder empezar a correr los threads de un warp diferente, del mismo bloque.
Si el bloque no cuenta con suficiente cantidad de threads para poner a correr de manera
concurrente, el SM va a forzosamente esperar que finalicen las operaciones de alta latencia
de estos warps sin nada que hacer mientras tanto. Si, por el contrario, se contasen con
miles de threads por bloque, entonces es posible que las operaciones que sirvan
para sincronizar los threads de todo un bloque en un punto espec\'ifico antes de proseguir
(un barrier) sean excesivamente costosas.

La arquitectura GPGPU de NVIDIA organiza los registros de todos los threads en un unico
register file, com\'un a todos los bloques. Como cada thread usa decenas de registros para guardar
los computos intermedios, NVIDIA decidio unficarlos, ya que es muy variable la cantidad que va a usar
cada kernel de ejecuci\'on. Una de las grandes diferencias entre Fermi y Kepler es la cantidad m\'axima de
registros por thread. Mientras que Fermi permitia hasta 63 registros, Kepler permite hasta 255. Esto
es una ventaja pero aumenta la presion sobre el register file. Cuando se lanzan muchos threads
que puedan estar corriendo paralelamente entre todos los SM de la GPU, es posible que se supere
la cantidad m\'axima de registros presentes en el register file. Esto fuerza a que el scheduler
no pueda poner a ejecutar m\'as bloques que los que pueda soportar este recurso, dejando SM ociosos.

Finalmente, al igual que con los registros, la memoria compartida es un recurso limitado. Como
solamente se cuenta con hasta 48Kb (Fermi-Kepler) de memoria de este tipo para ser repartida entre
todos los bloques que esten corriendo en todos los SM, el scheduler debera decidir no poner a ejecutar
m\'as bloques simultanemante que los que pueda soportar la cantidad de memoria compartida.

El problema tratado dentro de esta tesis cont\'o con todos estos limitantes. Afortunadamente,
las herramientas de profiling usadas remarca estos limitantes constamente, haciendolos
fundamentales a la hora de evaluar como proseguir en la busqueda de optimizaciones de c\'odigo.

\subsection{Cambios en el threading}
%Cambiamos de blocks por puntos, a blocks por funci\'on.
Originalmente el c\'odigo generabla un bloque por cada punto en el grupo de puntos
que teniamos que solucionar, con una sola dimension por thread.
Como este era el kernel que insumia el 94\% del tiempo de todo el proceso, decidimos
atacarlo de raiz, cambiando la paralelizaci\'on. A pesar de que el profiler de CUDA nos indicaba
que tenia una buena occupancy, decidimos cambiar de fondo el encare del problema.

El cuello de botella fundamental radicaba en como se distribuia el trabajo de computo
entre los kernels. Una cosa que se nota que este mecanismo de paralelizacion cuenta con
la ventaja de que casi no se compartian los datos entre los distintos threads, lo cual,
a priori, deberia haber ocasionado que no hubiera mucho mejor forma de realizar el computo.

Sin embargo, con un detalle m\'as fino, es posible notar que este mecanismo hacia que
muchos threads hicieran cuentas innecesarias, que no contribuian a la reducci\'on total.
Esto daba una alta occupancy ficticia, que buscamos acercar a la real. Adem\'as, a pesar
de que se compartian muy poca informaci\'on por la memoria compartida, tambien habia muchos
puntos de sincronización a nivel bloque.

XXXXXXXXX cambiamos a tener block\_height bloques por cada punto, y estos se operan
con una sola dimension de thread, donde block\_height es un parametro que se calcula
para cada grupo, y es el resultado de dividir la cantidad de funci\'ones que se van
a calcular por punto por la 2*DBS.

XXXXXXXXX Otro cambio importante que estudiamos con respecto al threading, es intentar realizar
mas de un punto por thread. Esto sirve para aprovechar un par de lecturas que son comunes
entre dos funciones. Con este cambio, se pueden ocultar algunas latencias de acceso, pero
cada grupo de threads lleva m\'as tiempo y usan m\'as registros. Esta estrategia es similar
a un loop unrolling manual, aplicado a la arquitectura GPU.

Un punto de intensa discusion es en el tema\~no de la particion de bloques para el calculo
de esta densidad. Para nuestro problema, decidimos utilizar una partici\'on por bloque
unidimensional, que sea multiplo de el tama\~no de un warp (32 threads). Esto permite
estudiar como afectan en el tiempo de procesamiento contar con uno o m\'as warps por
bloque. Una ventaja de usar bloques de 32 threads, es que el costo de la sincronizaci\'on
es exactamente cero. No se precisa sincronizar nada puesto que los threads trabajan
en lock-step sincronizados por warp. Un bloque chico ademas nos permite usar mas memoria compartida
por thread, dado que hay una cantidad fija de memoria por bloque (entre 32Kb y 64Kb).
Cuando se cuentan con muchos mas threads, se debe reducir este uso por thread de modo que
todos puedan ejecutar concurrentemente.

Dicho esto, la literatura \cite{NVIDIA_OPTIMIZATIONS} sugiere siempre que sea posible
usar bloques grandes y con threads lo m\'as independientes posibles. Una gran cantidad de threads
en un bloque permite tener muchos mas warps para schedulear de modo de esconder las latencias de
operaciones y de a accesos globales. Sin embargo, contar con muchos threads hace que las
sincronizaciones sean mucho m\'as costosas. Ademas, como cada SM no cuenta con pre-emption
de bloques, contar con muchos threads por bloque hace que los recursos se mantengan
por largos periodos.

Inicialmente, este tama\~no se habia fijado en 128 threads por bloque, 4 warps. Sin embargo,
al haber cambiado el threading a que realice dos cuentas por thread si puede, se aument\'o
el tiempo de ejecuci\'on de cada warp en promedio, pero menos que duplicar las cuentas.
Mostramos que tener solo un warp es bueno, pero mucho mejor a\'un es tener dos warps, porque
esto permite, con costo cero, poner a correr el otro para ocultar la latencia sin agrandar
demasiado el costo de la sincronizacion.


\subsection{Cambios en las memorias compartidas}
Accesos por bloque

\subsection{Cambios en los accesos globales}
La arquitectura de las placas de video estan pensadas entorno al poder de computo.
Las decisiones tomadas por los diseñadores de las GPGPU se concentran alrededor
de paralelismo a lo ancho, poniendo un gran enfasis en la cantidad de nucleos. Luego,
se dispone de menor cantidad de espacio disponible en el die para las memorias.

Esta decision implica que la amplia mayoria de la memoria de la GPGPU se encuentra
localizada externa al procesador.  No solo esta fisicamente mas lejos, sino que
adem\'as la latencia para accederla es altisima. Es decir, el paradigma de
programaci\'on de las GPGPU gira entorno a esconder la gran latencia de los accesos
a las memorias globales.

Una de las memorias intermedias entre entre el procesador y la memoria global es
la memoria de textura. La memoria de textura es un cach\'e sobre la memoria global,
que esta focalizado alrededor de los accesos a memoria en varias dimensiones.
Estas memorias reciben su nombre de su funci\'on principal, que es en el area de los
sistemas de video. Los mapas de textura suelen ser grandes matrices que definen
tanto los colores sobre las superficies de los poligonos como los relieves.
El detalle crucial de estas memorias es que un miss en estas cache, provoca
que se traigan datos no solo contiguos en memoria, como pasa en las caches de
CPU normalmente, sino que ademas se traigan los datos en posiciones logicas contiguas,
es decir, variando las distintas dimensiones de la matriz subyacente.

Las memorias de textura se ajustan bien a los problemas de GPGPU, porque se relacionan
intimamente con los los mecanismos de paralelismo de CUDA. Como los problemas se pueden
dividir en bloques con threads en $x$, $y$, $z$, entonces tiene mucho sentido pensar
que las estructuras de datos subyacentes se van a acceder usando indices multidimensionales.

En nuestro problema, la memoria de textura se presenta como una solucion para
los accesos bidimensionales de la matriz de RMM para el grupo de puntos.
Como esta matriz debe ser multiplicada por todos los valores de las funci\'ones,
derivadas primeras y segundas, se va a acceder a toda la matriz de RMM mas de
una vez por cada thread. Adem\'as, como se va a usar toda la matriz, y esta suele
tener un tamaño intermedio (es muy grande para memoria constante), el problema
suele entrar casi completamente en la memoria de textura.
La lectura bidimensional en este caso, se ajusta muy bien a los accesos por filas
y por columnas a la matriz.

El uso de la memoria de texturas agrega un recurso mas que tenemos que tener en
cuenta a la hora de profilear el c\'odigo. Para administrar los accesos a
la memoria de textura, cada multiprocesador tiene multiples "Unidades de textura".
Cuando dependemos de sobremanera de la memoria de textura para esconder la latencia,
se presentan contenciones sobre el acceso a las unidades de textura. Esto es
uno de los motivos por los cuales el procesador stallea los bloques hasta poder
ejecutarlos, cuando se liberan un poco mas los recursos.

%http://www.realworldtech.com/gt200/10/   << detalles sobre texture cache invalidation

\subsection{Cambios en la reducci\'on}
%Hubo que agregar reducci\'on de suma a nivel punto porque ya no se comparten mas la info

La unidad de GPGPU presenta grandes ventajas a la hora del computo masivamente distribuido.
Si logramos dividir el problema en pequeños kernels de ejecucion, podemos garantizar que
todos los multiprocesadores van a poder ejecutar constantemente grupos de threads, teniendo
siempre nuevos grupos listos para poder ponerlos a correr para poder esconder la latencia de uso.

El problema que se presenta al dividir tanto el computo, es que en algun momento se deberia juntar
todos los bloques independientes de ejecucion. Este proceso se conoce como reducci\'on, es decir,
aplicar una operacion a un vector de elementos tal que se obtenga un \'unico valor finalmente
como resultado de la cuenta.

En nuestro caso, al haber cambiado la paralelizacion de un bloque por punto, a un bloque
de funci\'ones, aparece una reducci\'on que antes no necesitabamos realizar. La reducci\'on que
requerida es una suma de todos los elementos de toda la columna que acabamos de procesar.
Como cada instancia del kernel pertenece a una columna de la matriz, y toda una columna se
procesa por un bloque, \textcolor{blue}{---REVISAR---}, entonces se puede compartir los datos
a nivel de bloque. Es decir, podemos hacer la reducci\'on reutilizando la memoria compartida
que usamos antes para pasarnos los distintos valores de cada fila de las funci\'ones y sus derivadas.

La reducci\'on eficiente es un tema ya muy estudiado en la bibliografia \textcolor{blue}{insertar cita a
reducci\'on y reducci\'on en cuda}. La reducci\'on en memoria compartida es fundamentalmente
una t\'ipica reducci\'on en \'arbol. Un detalle importante en este caso es que, a pesar de que
se reducen a lo sumo DENSITY\_BLOCK\_SIZE threads, es necesario hacer la reducci\'on en arbol
para poder maximizar la cantidad de threads ejecutando en paralelo la reducci\'on. Esta reducci\'on
binaria se realiza $\log_2{DBS}$ veces.

Este mecanismo de reducci\'on es generico y se puede emplear cualquier operacion binaria que
totalice las cuentas.


\subsection{Cambios en los pasajes de informaci\'on intrawarp}
%Los shuffles que no anduvieron salvo en function
La arquitectura SM35, junto con CUDA5, trajeron aparejadas una herramienta interesante
para el manejo interno de los pasajes de informaci\'on intra-warp durante la ejecuci\'on.
Las instrucciones de shuffle, como asi las denomina NVIDIA, son instrucciones que facilitan
el pasaje directo de un registro de un thread en un warp, a otro, en un solo ciclo de ejecuci\'on.
Estas instrucciones existen en diversas maneras, con distintos propositos. Principalmente se
utilizan para pasar de un thread al siguiente (modulo el warp size) un registro para poder seguir
operando. Otro uso que puede tener mucho interes proximamente son las instrucciones de votaci\'on,
donde se evalua un predicado para todos los threads, y se setea o limpia un bit en el resultado
de respuesta si se cumplio el predicado para ese thread. Con esta herramienta, no es necesario
acceder a memoria compartida para poder pasar minima informaci\'on dentro de cada warp.

Nuestro uso de las funci\'ones de shuffle consistio en intentar eliminar lo mas que podiamos
los accesos a la memoria compartida, fuente de bloqueos porque, a pesar de que ya corre
todos los bloques concurrentemente, leer elementos de ahi toman 4 ciclos en vez de uno solo
como en las funci\'ones de shuffle.

Probamos pasar de a un elemento y de a uno o dos vectores de 3 elementos, para comparar
cuan notable era el impacto del acceso mas veloz.

Finalmente concluimos que era una opcion valida para el pasaje de los valores de la funci\'on,
pero que el overhead de uso para cosas como los hessianos de la funci\'on no justifica el uso.
Adem\'as, como estas funci\'ones de shuffle solo estan presentes en las ultimas placas Kepler,
consideramos que el aprovechamiento marginal de los recursos no era lo suficientemente meritorio
de romper compatibilidad con las placas de la generacion Fermi anterior.






\subsection{Cambios en las operaciones matem\'aticas}
%Cambiar los vec\_type4 por 3 en los accesos a la shared es mucho mejor, no hace falta alinear ahi
Otro de los problemas existentes del c\'odigo que quisimos atacar fue el mejor empleo de las
memorias shared. Estas son un recurso finito y muy importante, ya que son un limitante de
la ocupancia de los multiprocesadores. Como pueden correr una cantidad de bloques que, a lo sumo,
no superen los 48Kb de memoria shared entre todos, es imprescindible minimizar el consumo real
de la memoria shared de modo que no estemos subutilizando los multiprocesadores.

Una cosa que probamos, con un grado de exito variante, fue disminuir el tamaño de los vectores
donde almacenamos las derivadas direccionales. Como nuestro problema es en tres dimensiones,
y los vectores estaban configurados para tener 4 valores por cada derivada, probamos llevarlos a
3, para que se ajusten a su consumo real de memoria. Este acercamiento no consider\'o el porque
se hizo asi de esta manera originalmente. Tener 4 valores consecutivos en memoria fuerza
al compilador a alinearlos a 16/32 bytes (simple y doble precision).
Esto presenta grandes ventajas a la hora de hacer transferencias de memoria en los accesos.

Sin embargo, este criterio no aplica a las memorias shared de la GPGPU. Como los accesos
a estas memorias se realizan de a 4 bytes y no de a 16/32, entonces no tiene ninguna ventaja
en particular realizar el alineamiento. Adicionalmente, como el cuarto valor no tiene forma de marcarse
como algo mas alla de padding, todavia se opera normalmente con el, por lo que nos ahorramos una
operacion de calculo. Mas a\'un, se presentan un ahorro del 25\% de los recursos de la memoria compartida
en contencion, sin ninguna desventaja a la hora de acceder a estos. Principalmente
esta mejora permite aumentar la cantidad de bloques corriendo concurrentemente en los multiprocesadores,
sin estas limitados por el uso individual de las memorias compartidas.
