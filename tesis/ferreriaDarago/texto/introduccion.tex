\subsection{Objetivos}

El principal objetivo de este trabajo es analizar distinas opciones dentro de
las ofertas de plataformas para HPC (\textit{High Performance Computing}) con
el prop\'osito de presentar puntos de \textit{trade-off} entre las mismas. Con la
intenci\'on de realizar esta comparaci\'on de una manera lo m\'as realista posible,
se utilizar\'a como caso de estudio un software de din\'amica molecular (LIO) que
ha sido probado en diversas ocasiones y sobre plataformas de \textit{hardware}
muy heterogéneas. Asimismo, nuestra comparaci\'on buscar\'on ser justa en t\'erminos
de la \textit{performance} obtenida para cada una de las arquitecturas
examinadas, pero sin dejar de tener en cuenta factores tan importantes a la
hora de afrontar un desarrollo de magnitud tal como lo es portar una
aplicación ya existente: la cantidad de c\'odigo que debe ser reescrito, la
mantenibilidad del c\'odigo resultante y la disponibilidad de recursos para
asistir en el proceso.

TODO MAL

\subsection{Arquitecturas estudiadas}

\subsubsection{Procesadores est\'andar}

Para servir de punto de comparaci\'on, en este trabajo tambi\'en analizaremos el 
comportamiento de arquitecturas como NVIDIA CUDA e Intel Xeon Phi con respecto a procesadores
CPU Estandar. En particular nos enfocaremos en la arquitectura Intel x86 de 64 bits.

Si bien las dem\'as arquitecturas analizadas difieren fuertemente de x86-64, estas diferencias
surgen de tendencias relativamente recientes en el desarrollo de procesadores que llevan a
tomar decisiones de \textit{tradeoffs} similares tambi\'en en procesadores m\'as usuales.

Una de estas tendencias es la disminuci\'on en el crecimiento de la \textit{performance} de
cada procesador o \textit{core} con respecto a a\~nos anteriores, empezando aproximadamente
desde 2003~\cite{HennessyPatterson}.

\begin{figure}[htbp]
    \centering
    \includegraphics[width=\textwidth]{images/processor-performance.jpg}
    \caption{Performance de procesador en los \textit{benchmarks} SPEC on respecto a VAX-11/780. Tomado de~\cite{HennessyPatterson}}
    \label{processor_performance}
\end{figure}

Las razones de esto son diversas, incluyendo limites f\'isicos de cantidad de transistores
por \textit{chip} por disipaci\'on de calor, pero tambi\'en involucran otros factores 
relacionados con los tipos de \textit{paralelismo} empleados.

A grandes rasgos, existen los siguientes tipos de paralelismo que puede aprovechar una
arquitectura para mejorar la \textit{performance} de ejecuci\'on:

\begin{itemize}
    \item \textit{Instruction Level Parallelism}: Este tipo de optimizaciones, que buscan
    ejecutar la mayor cantidad de instrucciones en un mismo hilo de ejecuci\'on simultaneamente,
    eran las m\'as usuales en los monoprocesadores. Optimizaciones de este estilo incluyen
    \textit{pipelines} de procesador para ejecutar multiples instrucciones de manera solapada,
    ejecuci\'on superescalar fuera de \'orden para ejecutar multiples instrucciones que 
    utilizan unidades del procesador distintas o que no dependen una de la otra, ejecuci\'on
    especulativa (basada en predicci\'on de saltos del procesador), etc. 

    \item \textit{Data Level Parallelism}: Consideran las optimizaciones cuyo prop\'osito es
    lograr aplicar una misma operaci\'on a cada elemento de un conjunto datos simultaneamente 
    en un mismo hilo de ejecuci\'on. Concierte por ejemplo a operaciones de vectorizaci\'on, 
    presentes en procesadores de la familia Intel mediante el set de operaciones SSE 
    (Streaming SIMD Extensions) o AVX (Advanced Vector eXtensions), que incluyen instrucciones
    para actuar sobre varios enteros o valores de punto flotante (por ejemplo, sumarlos, etc.)
    que estan puestos en un arreglo, al mismo tiempo. Por eso este modelo se denomina SIMD
    (Single Instruction, Multiple Data).

    \item \textit{Thread Level Parallelism}: Alejandonos de los procesadores monocore, tenemos
    la aparici\'on de m\'aquinas con m\'as de un procesador, que permiten por lo tanto hilos
    de ejecuci\'on totalmente independientes. Estos procesadores usualmente
    comparten la memoria principal (arquitectura SMP, \textit{Symmetric Multiprocessing}), lo
    cual conlleva a esfuerzo adicional para mantener consistencia y coherencia de la misma, y
    que no se convierta en un cuello de botella.  Adem\'as, existen optimizaciones por 
    \textit{hardware} para lograr multiples hilos de ejecuci\'on en un mismo procesador
    (por ejemplo Intel Hyperthreading).
\end{itemize}

La primera de estas maneras de explotar paralelismo fue la que prevaleci\'o mucho tiempo los
procesadores dominantes en el mercado (a pesar de que la taxonom\'ia de Flynn~\cite{HennessyPatterson} ya incluye 
procesadores que explotan todas ellas). La segunda y la tercera, base de arquitecturas como NVIDIA CUDA y Xeon Phi, 
est\'an tambi\'en haciendo aparici\'on en procesadores basados en x86: El procesador Intel Xeon E7-8800 posee 10 \textit{cores}
(2 hilos de ejecuci\'on por \textit{core}) a 2.4 GHz con un set de instrucciones SSE 4.2 que permite ejecutar una misma instrucci\'on
sobre 4 valores de punto flotante IEEE-754 de 32 bits (usando registros SSE de 128 bits)~\cite{XeonE78800Spec}. 

Estas tendencias en la busqueda de lograr m\'as hilos de ejecuci\'on y m\'as procesamiento simult\'aneo de datos, adem\'as
de mejoras al nivel instrucci\'on, tiene un impacto fuerte que diferencia sistemas pensados para computo intensivo de
sistemas de prop\'osito general. Este cambio de modelo requiere m\'as esfuerzo por parte del programador para aprovecharlo
correctamente, lo cual tambi\'en genera inter\'es en herramientas y literatura que sirva de gu\'ia al desarrollador, que
ahora no puede simplemente esperar a que la performance de un procesador se duplique cada 18 meses, y debe estructurar su
c\'odigo para hacer uso de los nuevos recursos que dispone.


\subsubsection{NVIDIA CUDA}

La siguiente arquitectura analizada en este trabajo es la arquitectura GPGPU desarrollada por NVIDIA, conocida
como CUDA por las siglas en ingles de \textit{Compute Unified Device Architecture}.
CUDA surge naturalmente de la aplicaci\'on de los pipelines desarrollados para
gr\'aficos, pero aplicados a computo cient\'ifico.

Las placas de video aparecen en 1978 con la introducci\'on de Intel del iSBX 275, permitiendo dibujar lineas,
arcos y bitmaps y comunicada por DMA al procesador principal. En 1985, la Commodore Amiga incluia un coprocesador
gr\'afico que podria ejecutar instrucciones independientemente del CPU, un paso importante en la separaci\'on
y especializaci\'on de las tareas. En la decada del 90, m\'ultiples
avances surgieron en la aceleraci\'on 2D para dibujar las interfaces gr\'aficas de los sistemas operativos,
y para mediados de la d\'ecada, muchos fabricantes estaban incursionando en las aceleradoras 3D como
add-ons a las placas gr\'aficas tradicionales 2D. A principios de la d\'ecada del 2000, se agregaron los
\textit{shaders} a las placas, peque\~nos programas independientes que corrian nativo en el GPU,
y se podian encadenar entre si, uno por pixel en la pantalla.~\cite{CG} Este paralelismo es el desarrollo fundamental
que llevaba a las GPU a poder procesar operaciones gr\'aficas ordenes de magnitud m\'as rapidas que el CPU.

En el 2006, NVIDIA introduce la arquitectura G80,
que es la primera placa de video que deja de resolver \'unicamente problemas especializados a gr\'aficos
para pasar a un motor gen\'erico donde cuenta con un set de instrucciones consistente para todos los
tipos de operaciones que realiza (geometria, vertex y pixel shaders) ~\cite{cudaHandbook}. Como subproducto de esto,
el GPU deja de tener pipelines especializados y pasa a tener procesadores sim\'etricos m\'as sencillos y m\'as
faciles de construir. Esta arquitectura es la que se ha mantenido y mejorado en el tiempo, permitiendo
a las GPU escalar masivamente en procesadores simples, de un bajo clock de una disipaci\'on t\'ermica
manejable.

Los puntos fuertes de las GPGPU modernas consisten en poder atacar los problemas de paralelismo
de manera pseudo-explicita, y con esto poder escalar ``facilmente'' si solamente se corre en una
placa mas r\'apida. ~\cite{} Te\'cnicamente esta arquitectura cuenta cientos a miles de procesadores
especializados en c\'alculo de punto flotante, procesando cada uno un \textit{thread} distinto pero
trabajando de manera sincr\'onica agrupados en bloques. Cada procesador a su vez cuenta con entre
64 a 256 registros ~\cite{NvidiaFermi}~\cite{NvidiaKepler}, como porci\'on de un register file de 64kb.
Las placas cuentas con m\'ultiples niveles de cach\'e y memorias especializadas (subproducto de
su dise\~no fundamental para gr\'aficos). Estos no poseen instrucciones SIMD, ya que su dise\~no primario
esta basado en cambio, en SIMT (\textit{Single Instruction Multiple Thread}), las cuales se ejecutan en los
bloques sincronicos de procesadores. De este modo, las placas modernas como la K40 alcanzan
poder de computo de 4.3 TFLOPs de precision simple, 1.7 TFLOPs de precision doble y 288GB/sec de
transferencia, usando 2880 CUDA Cores ~\cite{NvidiaKeplerDatasheet}. Para poner en escala la concentraci\'on
de poder de calculo, estas prestaciones harian de una computadora usando solo dos de estas placas
la supercomputadora m\'as potente del mundo en Noviembre 2001 ~\cite{Top500November2001}.

Para poder correr programas explotando la arquitectura CUDA, se deben escribir de manera que
el problema se particione usando el modelo de grilla de bloques de threads. Esto implica una
reescritura completa de los c\'odigos actuales en CPU y un cambio de paradigma importante, al
dejar de tener vectorizaci\'on, paralelizaci\'on automatica y otras t\'ecnicas tradicionales
de optimizaci\'on en CPU. Sin embargo, este trabajo ha rendido sus frutos en muchos casos:
en los \'ultimos 6 a\~nos, la literatura de HPC con aplicaciones en GPU ha explotado con
desarrollos nuevos basados en la aceleraci\'on de algoritmos num\'ericos (su principal uso).
% ~\cite{meter refs a gpu montecarlos}
Adem\'as, no todas las aplicaciones deben reescribirse de manera completa. Con la introducci\'on
de las librerias CuBLAS y CuFFT, se han buscado reemplazar con minimos cambios las historicas
librerias BLAS y FFTw, piedras fundamentales del computo HPC. ~\cite{cublas} ~\cite{cufft}.

Nuevas soluciones para la portabilidad se siguen desarrollando: las librerias como Thrust ~\cite{thrust},
OpenMP4.0 ~\cite{OpenMPspec} y OpenACC 2.0 ~\cite{OpenACCSpec} son herramientas que buscan hacer el
c\'odigo agnostico al acelerador de computo que usen. Estas permiten definir las operaciones de
manera gen\'erica y dejan el trabajo pesado al compilador para que subdivida el problema de la manera
que el acelerador (CPU, GPU, MIC) necesite. Obviamente, los ajustes finos siempre quedan pendiente para
el programador especializado, pero estas herramientas representan un avance fundamental al uso
m\'asivo de t\'ecnicas de paralelizaci\'on autom\'aticas, necesarias hoy dia y potencialmente
imprescindibles en el futuro.

La aplicaci\'on LIO ya contaba con una implementacio\'n CUDA desarrollada anteriormente a este
trabajo ~\cite{TesisNitsche}. En este trab\'ajo nos encargaremos de analizar algunos detalles internos de
la arquitectura CUDA usando esa implementaci\'on de referencia, y estudiar el impacto de las distintas
mejoras considerando los progresos que han surgido en las iteraciones de CUDA desde entonces.



\subsubsection{Intel Xeon Phi}

La última de las arquitecturas examinadas en este trabajo es un desarrollo relativamente reciente
por parte de Intel, el coprocesador Xeon Phi. Desde su introduccion al mercado en el a\~no 2012, se ha visto implementaciones de gran magnitud con esta tecnología, como puede verse en el papel
que juega dentro de supercomputadoras como la Tianhe-2 de la Universidad de Sun Yat-Sen en China,
listada en TOP 500 como la supercomputadora más rápida del mundo en Junio 2013, Noviembre 2013 y
Junio 2014. Esta supercomputadora de 16000 nodos, cada uno de los cuales consiste de dos Intel
Ivy Bridge Xeon en combinación con 3 coprocesadores Xeon Phi, es teóricamente capaz de alcanzar los
54.9 petaflops de computo, y superó por casi el doble de flops a su competidor más cercano en
Junio de 2013.

Los principales puntos fuertes de esta nueva arquitectura no son solo sus especificaciones tecnicas, que prometen una
capacidad de computo de 1 TFLOPs y 240 GB/s de transferencia de memoria, sino que además ha de considerarse
que, a diferencia de sus competidores directos NVIDIA CUDA y OpenCL, el esfuerzo de portar la aplicación a utilizar
el Xeon Phi es presentado como mucho menor por parte de Intel. Esta combinación de performance y portabilidad
presenta al Xeon Phi como una alternativa muy interesante en el campo de aplicaciones intensivas en cómputo.

Para lograr esto, Intel desarroll\'o una nueva arquitectura de muchos procesadores (Many Integrated Core Architecture, MIC) de
tipo SMP (Symmetric MultiProcessing), bajo el nombre en código \textit{Knights Corner} y similar a desarrollos anteriores como \textit{Larrabee}. El coprocesador dispone de 60 cores basados
en el dise\~no del Intel Pentium, con una ISA (Instruction Set Architecture) muy similar al set de instrucciones IA-32, mas no compatible con el mismo. Además de su cantidad
numérica, estos cores disponen de registros SIMD (Single Instruction Multiple Data) de 512 bits (el doble de tamaño que los registros de 256
bits que disponen los procesadores de Intel existentes en el set de instrucciones AVX de x86). Esto implica que programas
que ya explotan fuertemente el paralelismo a nivel datos y a nivel \textit{threads} de computo debieran poder hacer uso de
estas nuevas prestaciones con una recompilación del c\'odigo de la aplicaci\'on. Adicionalmente el coprocesador puede usarse para que el
\textit{host} delege calculos intensivos en el mismo (con el cual esta conectado por un bus
PCI Express de alta velocidad) y luego obtenga los resultados del mismo. Este patr\'on de uso es similar a los aceleradores de computo como las GPGPU modernas. Ambos modos son soportados por el \textit{toolchain} que provee
Intel para desarrollo de software, que consiste no solo de compiladores de C, C++ y Fortran, sino que también de librerías especialmente optimizadas para uso del Xeon Phi
como lo es la Intel MKL (Math Kernel Library).

Sin embargo, factores diversos hacen que si bien el esfuerzo de portar una aplicación ya existente para aprovechar el Xeon Phi es menor en principio
a otras opciones como NVIDIA CUDA, el mismo no es insignificante. La cantidad de cores se ve compensada por su relativa baja velocidad de clock
(aproximadamente 1.0 GHz), lo cual requiere mas esfuerzo por parte del programador para hacer buen uso de la escalabilidad tanto en cores como en
paralelismo en uso de los datos (vectorización), entre otros aspectos.

Los detalles particulares de esta arquitectura y su impacto serán descriptos y analizados en capítulos posteriores.
La literatura con respecto a esta plataforma en Q3 de 2014 es existente, aunque aun escasa~\cite{Fang}. Estos trabajos
exploratorios buscan analizar distintos aspectos del uso del coprocesador a trav\'es de  modelos sencillos (microbenchmarks, optimizar aplicaciones de peque\~no tama\~no). En este trabajo buscamos
poner las recetas y consejos obtenidos para el uso del Xeon Phi en contexto de tecnolog\'ias competidoras ya establecidas, dentro del marco de portar una aplicaci\'on cient\'ifica de mediano tama\~no ya en uso.


\subsection{Problema abordado}

El software que servirá como caso de estudio para este trabajo es un software de 
din\'amica molecular denominado LIO. Este software ha sido objeto de otros 
trabajos de investigación~\cite{PaperNitscheManu}~\cite{TesisNitsche} y ha sido 
usado en trabajos de simulaci\'on en din\'amica molecular.

A continuaci\'on presentaremos un breve res\'umen de los aspectos te\'oricos detr\'as de
la herramienta, de manera de servir de base para entender las distintas partes optimizadas
del mismo.

El prop\'osito de este \textit{software} es realizar simulaciones de las propiedaes
de sistemas qu\'imicos complejos (solutos en soluci\'on, prote\'inas, etc.). Estas
propiedades est\'an estudiadas dentro del marco te\'orico de la mec\'anica cu\'antica
modelando el sistema como un conjunto de part\'iculas.

En principio todas las propiedades de un sistema de part\'iculas livianas
(como los electr\'ones) pueden ser obtenidas a partir de la funci\'on de onda $\Psi$,
la cual obedece la ecuaci\'on de onda de Schr\"odinger \textit{dependiente del tiempo},

\begin{equation}
    \label{schro_time_dep}
    -\hbar\frac{\partial \Psi}{\partial t} (\mathbf{r},t) = \frac{-\hbar^2}{2\mu}\nabla^2 \Psi(\mathbf{r},t) + V(\mathbf{r},t) \Psi(\mathbf{r},t)
\end{equation}

donde $\mathbf{r} = (r_1,\dots,r_n)$ es el vector de todas las posiciones de las part\'iculas del sistema,
$m$ es la masa de una part\'icula cualquiera, $V$ es un campo externo que afecta a las part\'iculas y
$\hbar$ es la constante de Planck divida por $2\pi$. En esta versi\'on, el campo $V$ depende del tiempo; si
esto no ocurre se puede simplificar utilizando el operador Hamiltoniano

\begin{equation*}
    \hat{H} =  -\frac{\hbar^2}{m} \nabla^2 + \hat{V}
\end{equation*}

a la ecuaci\'on de Sch\"odinger \textit{independiente del tiempo}

\begin{equation}
    \label{schro_time_indep}
    \hat{H} \Psi(\mathbf{r}) = E \Psi(\mathbf{r})
\end{equation}

Donde $E$ es la energ\'ia asociada a la funci\'on de onda $\Psi$.

Ahora, resolver esta ecuaci\'on diferencial no se puede hacer de manera exacta para sistemas que involucran m\'as
de una part\'icula, por lo cual se utilizan m\'etodos aproximados. Estos m\'etodos son computacionalmente muy caros
(del \'orden c\'ubico o mayor) por lo cual otro enfoque es necesario para resolver el problema.

El enfoque de LIO esta dentro del marco de t\'ecnicas de simulaci\'on h\'ibrida denominadas QM/MM 
(\textit{Quantum Mechanical}/\textit{Molecular Mechanical}). Este enfoque consiste en resolver la ecuaci\'on de
Schr\"odinger en la porci\'on activa del sistema (por ejemplo el soluto) y modelar el resto (el solvente por
ejemplo) mediante din\'amica cl\'asica. De esta manera se obtiene, con un costo moderado, el comportamiento de
la porci\'on activa modulado por el comportamiento del resto del sistema.

Para resolver num\'ericamente~\ref{schro_time_indep} en la porci\'on activa del sistema se utiliza el m\'etodo
de DFT (\textit{Density Functional Theory}). La base de este m\'etodo consiste en la introducci\'on de la
\textit{densidad electr\'onica} $\rho$, que viene a representar la probabilidad de encontrar un electr\'on en
el espacio dado un estado de la configuraci\'on del sistema. $\rho$ y $\Psi$ (y por lo tanto $E$) se encuentran relacionadas por

\begin{equation}
    \label{honenberg_kohn}
    \rho(\vec{r}_i) = \int \Psi^{\dagger}(\vec{r}_1, \dots, \vec{r}_n) \Psi(\vec{r}_1, \dots, \vec{r}_n) d\vec{r}_1 \dots d\vec{r}_n \qquad i \in [1,n]
\end{equation}

Donde $\Psi^{\dagger}$ es el conjugado de $\Psi$. La energ\'ia entonces es un \textit{funcional} de la densidad, $E[\rho]$. Otros trabajos de DFT dan m\'as
detalles a esta relaci\'on. Uno de los principales resultados de DFT es la siguiente ecuaci\'on para la relaci\'on entre $E$ y $\rho$:

\begin{equation}
    \label{honenberg_kohn} 
    E[\rho] = T_s[\rho] + V_{ne}[\rho] + \frac{1}{2} \int \int \frac{\rho(\vec{r}_1) \rho(\vec{r}_2)}{r_{12}} d\vec{r}_1 d\vec{r_2} + E_{xc}[\rho]
\end{equation}

Donde $T_s[\rho]$ es la energ\'ia cin\'etica asociada con la densidad, $V_{ne}[\rho]$ es la energ\'ia potencial producto de la interacci\'on entre los
electr\'ones (la densidad) y los n\'ucleos, el tercer t\'ermino es el resultado de la repulsi\'on de Coulomb entre eletrones y $E_{xc}[\rho$ es la
energ\'ia de intercambio y correlacci\'on.

Para este trabajo, debido a su costo computacional~\cite{PaperNitscheManu}, nos interesar\'a sobre todo el computo de la energ\'ia de intercambio y correlacci\'on ($E_{xc}$).
Esta energ\'ia es calculada mediante 

\begin{equation}
    E_{XC} = \int \rho(r) \epsilon_{xc}(\rho(r)) dr
\end{equation}

que es integrada numericamente en una grilla, mediante la suma

\begin{equation}
    E_{XC} \approx \sum_j \rho(r_j) \epsilon_{xc} (\rho(r_j))
\end{equation}

TODO: CONTINUAR ESTO.

