\section{Aceleraciones alcanzadas}
%
%\begin{figure}[htbp]
%   \centering
%   \includegraphics[width=\plotwidth]{plots/cuda/final.png}
%   \caption{Speedup en veces de correr Hemoglobina comparando implementaci\'on original en CUDA contra
%   las versiones optimizadas definitivas}
%   \label{plt:cuda-final}
%\end{figure}


\section{?`Qu\'e me conviene comprar?}


\section{Importancia del problema}
\section{Impacto de trabajar correctamente con la arquitectura}
\section{Posible trabajo a futuro}
De este trabajo pudimos observar al menos cinco categorias de problemas que se
desprenden para poder extender LIO y estudiar posibles mejoras performance considerando
las arquitecturas disponibles en el mercado.
\subsection{Versiones h\'ibridas}
\begin{enumerate}
  \item Hacer una versi\'on h\'ibrida CPU-GPU.
  \item Hacer una versi\'on h\'ibrida CPU-XeonPhi con offloading.
  \item Hacer una versi\'on h\'ibrida CPU-GPU-XeonPhi
  \item Probar implementar en FPGA los c\'alculos de SCF para poder resolverlos por hardware en procesadores
    Atom.
  \item Implementar una versi\'on MPI para poder resolver una iteraci\'on distribuyendo
    a m\'ultiples CPU/GPU/XeonPhi en distintos nodos.
  \item Explotar paralelismo de etapas a nivel mas granular, como realizar las densidades en CPU y las matrices
    de Kohn-Sham en GPU.
\end{enumerate}

\subsection{Balance de cargas}
\begin{enumerate}
  \item Probar si vale la pena escalar a m\'ultiples XeonPhi.
  \item Repensar el algoritmo de partici\'on de trabajos para hacer m\'as equitativas las cargas sin
    tener que recurrir al balanceo durante la iteraci\'on.
  \item Modificar el algoritmo de generaci\'on de grilla que genere grupos m\'as equitativos para poder
    distribuir mejor las cargas.
  \item Estudiar el problema de partici\'on y sus funciones de costos para sean acertadas tanto en CPU
    como en GPU.
\end{enumerate}

\subsection{Explotar m\'as paralelismo}
\begin{enumerate}
  \item Portear LIO a OpenCL para poder unificar el c\'odigo.
  \item Investigar el uso de librer\'ias BLAS (Magma, MKL, CUBLAS, ATLAS, etc.) para offloadear fragmentos del
    c\'alculo de SCF.
  \item Analizar la posibilidad de usar CUDA Streams para intentar lograr kernels concurrentes y maximizar
    el uso de una placa.
  \item Acelerar los pasos de SCF actualmente single-core y no en CPU.
  \item Acelerar el c\'alculo de las contribuciones de Coulomb para las fuerzas \'inter at\'omicas.
  \item Analizar otras estrategias de paralelismo en sistemas distribuidos como MapReduce para sistemas muy grandes.
  \item Estudiar alg\'un m\'etodo para elegir en automaticamente CPU cuando particionar a nivel
    de m\'ultiples grupos y cuando a nivel interno por grupo.
\end{enumerate}

\subsection{Estudio autom\'atico de par\'ametros}
\begin{enumerate}
  \item Hacer un algoritmo adaptativo que busque la mejor configuraci\'on de par\'ametros de tama\~no de cubos y esferas.
  \item Estudiar como paralelizar y vectorizar functions en CPU para poder escalar a m\'ultiples cores.
  \item Hacer alg\'un m\'etodo de aprendizaje autom\'atico para hacer un pretuneo de ciertas constantes en GPU como
    BLOCK\_SIZE para evitar tener que reajustar en cada nueva generaci\'on de dispositivos.
  \item Analizar el consumo energ\'etico de la aplicaci\'on e intentar buscar otras arquitecturas para maximizar
    la eficiencia por Watt.
\end{enumerate}

\subsection{Nuevas aplicaciones qu\'imicas}
\begin{enumerate}
  \item Experimentar el comportamiento de sistemas muy grandes que no entren en memoria; como fraccionarlos.
  \item Integrar LIO a otros sistemas de MM como GROMACS y CHARMM.
\end{enumerate}
