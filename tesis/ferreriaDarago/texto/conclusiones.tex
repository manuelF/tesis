%\section{Importancia del problema}
%\section{Impacto de trabajar correctamente con la arquitectura}
\section{Posible trabajo a futuro}
El trabajo pendiente inmediato que se ve ac\'a, que brindar\'ia inicialmente la mayor aceleraci\'on
es la realizaci\'on de una versi\'on h\'ibrida de LIO. Como pudimos observar, la resoluci\'on de grupos es
totalmente independiente entre si (salvo una reducci\'on final). Esto se explot\'o en forma de paralelismo
de m\'ultiples placas GPU y en m\'ultiples n\'ucleos de procesador. Nada, \textit{a priori}, previene
que se pueda usar simult\'aneamente como recursos asim\'etricos para resolver de forma h\'ibrida los sistemas.

La exploraci\'on preliminar que se realizo con el Xeon-Phi en este trabajo no trajo los resultados
prometidos por las especificaciones del fabricante. La velocidad de los cores y la de la memoria
presentaron cuellos de botella fundamentales que, al menos durante esta investigaci\'on, no se
pudieron resolver. Una versi\'on h\'ibrida con offloading es prometedora, pero las aceleraciones m\'aximas
estar\'an limitadas por los costos de transferencia de memoria entre CPU y el Xeon Phi.

Un \'area de inter\'es para estudiar en la aplicaci\'on es la construcci\'on de las particiones
con los mejores par\'ametros para cada configuraci\'on. Si la aplicaci\'on pudiera descubrirlos
independientemente de tener que cargarlos a mano, seria un gran progreso para evitar mucho
tiempo de ajuste fino manual. Esto tambi\'en se puede expandir para muchos de los par\'ametros
que se optimizar en GPU (el tama\~no de los bloques de todos los kernels) y en CPU (TODO).
Esto permitir\'ia que, simplemente con una recompilaci\'on del c\'odigo, se pueda correr en generaciones
pr\'oximas de GPU y CPU, sin tener que reescribir estas secciones de c\'odigo cada 2 o 3 a\~nos.

Adem\'as de estas anteriores, pudimos observar al menos cinco categor\'ias de trabajos futuros que se
desprenden para poder extender LIO y estudiar posibles mejoras performance considerando
las arquitecturas disponibles en el mercado.
\subsection{Versiones h\'ibridas}
\begin{enumerate}
 \item Hacer una versi\'on h\'ibrida CPU-GPU-XeonPhi
  \item Probar implementar en FPGA los c\'alculos de SCF para poder resolverlos por hardware en procesadores
    Atom.
  \item Implementar una versi\'on MPI para poder resolver una iteraci\'on distribuyendo
    a m\'ultiples CPU/GPU/XeonPhi en distintos nodos.
  \item Explotar paralelismo de etapas a nivel mas granular, como realizar las densidades en CPU y las matrices
    de Kohn-Sham en GPU.
\end{enumerate}

\subsection{Balance de cargas}
\begin{enumerate}
  \item Repensar el algoritmo de partici\'on de trabajos para hacer m\'as equitativas las cargas sin
    tener que recurrir al balanceo durante la iteraci\'on.
  \item Modificar el algoritmo de generaci\'on de grilla que genere grupos m\'as equitativos para poder
    distribuir mejor las cargas.
  \item Estudiar el problema de partici\'on y sus funciones de costos para sean acertadas tanto en CPU
    como en GPU.
\end{enumerate}

\subsection{Explotar m\'as paralelismo}
\begin{enumerate}
  \item Portear LIO a OpenCL para poder unificar el c\'odigo.
  \item Investigar el uso de librer\'ias BLAS (Magma, MKL, CUBLAS, ATLAS, etc.) para offloadear fragmentos del
    c\'alculo de SCF.
  \item Analizar la posibilidad de usar CUDA Streams para intentar lograr kernels concurrentes y maximizar
    el uso de una placa.
  \item Acelerar los pasos de SCF actualmente single-core y que no tienen implementaciones en GPU.
  \item Acelerar el c\'alculo de las contribuciones de Coulomb para las fuerzas \'inter at\'omicas, el mayor factor de SCF luego de XC.
  \item Analizar otras estrategias de paralelismo en sistemas distribuidos como MapReduce para sistemas muy grandes.
  \item Estudiar alg\'un m\'etodo para elegir autom\'aticamente en CPU cuando particionar a nivel
    de m\'ultiples grupos y cuando a nivel interno por grupo.
\end{enumerate}

\subsection{Estudio autom\'atico de par\'ametros}
\begin{enumerate}
  \item Estudiar como paralelizar y vectorizar el calculo de funciones en CPU para poder escalar a m\'ultiples cores.
  \item Analizar el consumo energ\'etico de la aplicaci\'on e intentar buscar otras arquitecturas para maximizar
    la eficiencia por Watt.
\end{enumerate}

\subsection{Nuevas aplicaciones qu\'imicas}
\begin{enumerate}
  \item Experimentar el comportamiento de sistemas qu\'imicos muy grandes que no entren en memoria; como fraccionarlos.
  \item Integrar LIO a otros sistemas de MM como GROMACS y CHARMM.
\end{enumerate}
