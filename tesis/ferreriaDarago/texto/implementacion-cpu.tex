Asi como ya exist\'ia una base de c\'odigo original de CUDA, el
trabajo sobre CPU tambi\'en se realiz\'o sobre el c\'odigo original
existente. El mismo fue dise\~nado de manera uniprocesador, y teniendo
en mente un set de instrucciones SIMD anterior a AVX, donde el tama\~no
de los registros de procesador era de 128 bits.

Con el prop\'osito de adaptar el c\'odigo para procesadores
paralelos y vectoriales como lo son los de la gama Xeon y Xeon Phi de
Intel, se busc\'o vectorizar y paralelizar el c\'odigo tanto como fuese
posible. En particular, se prioriz\'o lograr una gran escalabilidad en
n\'umero de procesadores, especialmente en las pruebas realizadas en el
Xeon.

De acuerdo a la bibliografi\'ia~\cite{Jeffers}, es necesario lograr que
el c\'odigo este no solamente bien vectorizado sino que escale con la
cantidad de procesadores para poder hacer uso de las prestaciones del
coprocesador Xeon Phi. Por lo tanto, las pruebas iniciales se concentraron
en lograr buena escalabilidad y vectorizac\'on en CPU.

Puesto que muchas decisiones arquitecturales del c\'odigo se realizaron en
base a experimentos con prototipos representativos de las diversas operaciones,
incluiremos algunos detalles del dispositivo utilizado para las pruebas.

La computadora utilizada para las pruebas fue un servidor \textit{dual-socket}
con 12 procesadores Intel Xeon CPU E5-2620 v2, dividos en dos grupos de 6. Los
procesadores corren a una frecuencia de clock de 2.10 GHz, y soportan el set
de instrucciones x86-64 con AVX 1.
Cada procesador cuenta con 64 Kb de cache L1, 2 Mb de cache L2 para cada par de cores, y
15 Mb de cache L3 compartida.
El mismo contaba con 32 GB de memoria RAM DDR3 a 4 canales de memoria, dando 
una transferencia te\'orica m\'axima de 42.6 GB/s, con ciclo de clock a 1.3 MHz.

La familia de procesadores Xeon cuenta con tecnolog\'ia Turbo Boost. La misma 
incluye una funcionalidad en el chip que ajusta din\'amicamente la frecuencia
de los procesadores de acuerdo a la temperatura y potencia empleadas. Esto, si
bien a \textit{a priori} es algo \'util ya que mejora la performance de los 
programas, dificulta el an\'alisis de escalabilidad ya que al utilizar m\'as
procesadores, aumenta el consumo energ\'etico y temperatura y por tanto la
frecuencia de los procesadores disminuye. Para evitar este efecto en nuestro
an\'alisis se deshabilito Turbo Boost al realizar las pruebas.

Asimismo, originalmente los procesadores Intel Xeon cuentan con Hyperthreading,
dando un total de 24 procesadores virtuales en vez de 12 n\'ucleos f\'isicos.
Sin embargo, los dos hilos de ejecuci\'on (\textit{hyperthreads}) en un mismo
procesador comparten unidades b\'asicas como las ALU. Al no ser totalmente
independientes, esto tambi\'en dificulta el an\'alisis. Para no tomar esto en
cuenta se trabajo con Hyperthreading deshabilitado.

Las pruebas tambi\'en se ajustaron en base al coprocesador Xeon Phi que estaba
conectado al Xeon como host. El modelo de coprocesador usado contaba con 61 
cores y 8 Gb de memoria RAM, los valores est\'andar para la l\'inea actual de
Xeon Phi.

La secci\'on de c\'odigo trabajada corresponde a la parte del procesamiento
de LIO optimizada mediante CUDA. Esta parte del c\'odigo estaba ya implementada
en C++, utilizando extensiones de Intel ICC para vectorizaci\'on . No se busc\'o
optimizar otras secciones de c\'odigo ya que las mismas no contaban con una
contrapartida en CUDA, haciendo entonces imposible comparar esta arquitectura con
Intel Xeon y Xeon Phi. Si bien el tiempo de ejecuci\'on de esas secciones, por su
caracter serial y poco optimizado, terminan consumiendo una fracci\'on del tiempo
no despreciable, se consider\'o fuera del enfoque de este trabajo trabajar sobre
ellas. La optimizaci\'on de las mismas para aprovechar de las arquitecturas
estudiadas queda como trabajo a futuro.

\subsection{Estructura original del c\'odigo}

Un esquema de alto nivel del computo m\'as intensivo realizado por el paquete
G2G se encuentra en la figura~\ref{algo:lio-iteration}. Este c\'omputo corresponde
a una iteraci\'on de c\'omputo de la matriz de Kohn-Sham. El pseudoc\'odigo
corresponde a la implementaci\'on en C++ del mismo para los casos donde se
utiliza GGA (\textit{Gradient Global Approximation}) y se calculan tanto las
fuerzas como la matriz resultado.

El marco general de la aplicaci\'on a nivel pseudoc\'odigo se encuentra en la
figura~\ref{algo:lio-general-schematics}. Corresponde al detalle del esquema
a alto nivel mostrado en la figura~\ref{fig:lio-steps}.

Las matrices $G$ y $H$ corresponden a matrices de vectores $\mathbb{R}^3$, y
la matriz $F$ de valores de funciones es de escalares. Las operaciones entre
vectores y vectores y escalares tienen la sem\'antica esperable de algebra
lineal. El producto entre dos vectores debe ser interpretado como producto
componente a componente, no como producto escalar o vectorial.

La implementaci\'on de las operaciones entre vectores merece particular atenci\'on.
Para aprovechar el set de instrucciones SSE 4, la ultima versi\'on disponible al
momento de realizar la implementaci\'on, la clase que implementa un vector de 3
componentes (\texttt{cvector3}) se adapt\'o mediante herencia para mapear a un
registro de SSE 4. Dado que el ancho de estos registros es de 128 bits, los mismos
permiten realizar calculos de a 4 punto flotante a la vez. Al ser 3, uno de los
elementos del registro SSE no se utiliza.

La representaci\'on de un vector de tres componentes de esta manera, si bien da
un \textit{speedup} significativo, no es portable ni escalable. Al incrementar el
ancho de registro SIMD m\'as de los campos deben ser ignorados, malgastando cada
vez m\'as espacio. Asimismo, se desperdician oportunidades por parte del compilador
para optimizar mejor haciendo uso de todos los registros y operaciones que dispone
la arquitectura. 

Uno de los computos, \texttt{ssyr}, utilizaba la GSL (\textit{GNU Scientific Library}).
La misma se reemplazo por la MKL (\textit{Math Kernel Library}) de Intel ya que
la operaci\'on corresponde a una subrutina de BLAS nivel 2 muy conocida. Esto se
realiz\'o principalmente para evitar tener que hacer una compilaci\'on y linkeo
de la librer\'ia GSL en el Xeon Phi, y para disminuir la cantidad de c\'odigo a
optimizar dado que la MKL ya se utilizaba en otras secciones del programa.

\begin{algorithm}[H]
        \caption{Pseudoc\'odigo de la iteraci\'on original de LIO}
        \label{algo:lio-iteration}
        \begin{algorithmic}
            \Function{iteration}{$PG : PointGroup, Forces: \mathbb{R}^{fr \times fc}, RMMO: \mathbb{R}^{rr \times rc}$}
              \State $RMM^{IN} \gets rmm\_input(G)$
              \State $F \gets functions(PG)$
              \State $G \gets gradient(PG)$
              \State $H \gets hessian(PG)$
              \State $energy \gets 0$
              \State $n,m \gets \# functions(PG), \# points(G)$
              \ForAll{$p \in [0..m)$}
                  \State $pd, dxyz,dd1,dd2 \gets 0, (0,0,0), (0,0,0), (0,0,0)$
                  \ForAll{$i \in [0..n)$}
                      \State $w \gets \Sigma_{0 \leq j \leq i} F_{p,j} \cdot RMM_{ij}$
                      \State $w3 \gets \Sigma_{0 \leq j \leq i} G_{p,j} \cdot RMM_{ij}$
                      \State $ww1 \gets \Sigma_{0 \leq j \leq i} H_{p,2j} \cdot RMM_{ij}$
                      \State $ww2 \gets \Sigma_{0 \leq j \leq i} H_{p,2j+1} \cdot RMM_{ij}$
                      \State $pd \gets pd + F_{p,i} w$
                      \State $dxyz \gets dxyz + G_{p,i} \cdot w + w3 \cdot F_{p,i}$
                      \State $dd1 \gets dd1 + 2 \cdot w3 \cdot G_{p,i} + H_{p,2i} \cdot w + ww1 \cdot F_{p,i}$
                      \State $FgXXY \gets (G_{p,i}^x, G_{p,i}^x, G_{p,i}^y)$
                      \State $w3XXY \gets (w3^x, w3^x, w3^y)$
                      \State $w3YYZ \gets (w3^y, w3^y, w3^z)$
                      \State $FgYZZ \gets (G_{p,i}^y, G_{p,i}^z, G_{p,i}^z)$
                      \State $dd2 \gets dd2 + FgXXY \cdot w3YYZ + FgYYZ \cdot w3XXY + H_{p,2j+1} \cdot w + ww2 \cdot F_{p,i}$
                  \EndFor 
                  \State $exch, corr, y2a \gets potencial(pd, dxyz, dd1, dd2)$
                  \State $energy \gets energy + (weight(p) \cdot pd) \cdot (exch + corr)$
                  \State $FR_p \gets y2a \cdot weight(p)$
              \EndFor
              \State $RMM^{OUT}_{i,j} \gets \Sigma_{k < m} F_k \cdot F_k' \cdot FR_k$
              \ForAll{$i,j \gets rmm\_indexes(PG)$}
                  \State $RMMO_{i,j} \gets RMMO_{i,j} + RMM^{OUT}_{i,j}$
              \EndFor
            \EndFunction
        \end{algorithmic}
\end{algorithm}

\subsection{Cambios en la vectorizaci\'on}


\subsection{Prototipo usando MKL}

\subsection{Prototipos de paralelizacion}

\subsection{Modificaciones en el algoritmo}

\subsection{An\'alisis de cargas}

\subsection{Cambios en paralelismo}

\subsection{Algoritmo de particionado}

\subsection{Algoritmo de balanceo}
