\subsection{Arquitectura y features CUDA}

La arquitectura CUDA esta enfocada a procesamiento de grandes cantidades de datos
de puntos flotante. El procesador GPGPU cuenta con cientos de ALU sincronizadas
por bloques, permitiendo un paralelismo adaptativo a distintos problemas.

El procesamiento GPGPU es similar al procesamiento vectorial
realizado por las supercomputadoras Cray y IBM que surgio en los 1960's, pero
en vez de usar VLIW para procesamiento masivo, CUDA usa multiples hilos de ejecuci\'on
que trabajan simultaneamente sobre los datos.
El procesamiento consiste en un funcionamiento hibrido entre compilador y procesador. Se determina
un conjunto de elementos a procesar y se elije de manera explicita una manera de particionar el
problema a la hora de ser enviado para procesado a la placa.

El paralelismo en CUDA principalmente se divide en varios niveles:
\begin{itemize}
  \item Block Level
  \item Grid Level
  \item Board level
\end{itemize}

El paralelismo a nivel de block agrupa de a bloques 32 threads (llamados \texttt{warp} en CUDA).
Cada uno de estos threads va a contar con el mismo blockId y su propio threadId, identificandolos univocamente.
Estos threads van a correr simultaneamente en el mismo SM (\texttt{Streaming Multiprocessor}) y
van a ser puestos y sacados de ejecuci\'on dinamicamente por el scheduler de hardware que cuenta la GPU.

El paralelismo a nivel de grid determina un arreglo de bloques de ejecuci\'on que particiona
el dominio del problema. El block scheduler va a ejecutar los distintos bloques a ejecutar hasta
el final de la ejecuci\'on de todos los threads. Cuando este bloque termine, y si se cuentan con
los recursos apropiados, se pondr\'a a correr el siguiente bloque disponible.

El paralelismo a nivel de placa consiste en poder distribuir la carga entre distintas
GPGPU dispuestas en un mismo sistema compartiendo una memoria RAM com\'un. CUDA no cuenta con un
modelo implicito de paralelismo entre distintas placas, pero es posible hacerlo manualmente.

Para realizar el computo, esta arquitectura cuent a su vez con multiples clases de memorias
que se adaptan de maneras diferentes a los distintos procesos. Estas incluyen:

\begin{itemize}
  \item Memoria global
  \item Registros
  \item Memoria local
  \item Memoria compartida
\end{itemize}

La memoria global es la memoria principal fuera del chip del GPGPU. Esta es de gran tama\~no
y es compartida por todos los procesadores de la GPGPU y los CPU que integran el
sistema. Es decir, tanto los GPGPU y los CPU pueden invocar las funciones del driver CUDA
para poder transferir datos entre la placa y la memoria RAM. Esta memoria es extremadamente
lenta en comparaci\'on con el procesador, por lo cual se debe siempre intentar recurrir
a las demas memoria que componen la jerarquia del GPGPU antes que acceder seguido a la memoria
global.


Los registros son la unidad basica de almacenamiento de los kernels de ejecuci\'on.
Cada thread de ejecuci\'on cuenta con una cantidad limitada de registros de punto flotante de
32 bits con latencia de un par de ciclos de clock. A su vez, existen una cantidad finita de
registros totales que cuenta un chip (oscilan entre 16.535 y 65.535 registros).

La memoria compartida es una memoria que es visible para todos los threads dentro
de un bloque, pero es independiente entre bloques. Cada thread puede escribir en cualquier
parte de la memoria compartida de su bloque y puede ser leido por cualquier otro thread
de su bloque. Es una memoria muy r\'apida, que tarda una peque\~na cantidad de ciclos
de acceso de lectura. Esta memoria es compartida con la cach\'e nivel 1. Es limitada en
tama\~no y cuenta con una capacidad de entre 32Kb y 64kb que debe ser compartida por todos los bloques.

La memoria local es una memoria propia de cada thread, que se encuentra localizada dentro de la
memoria global. Esta memoria es usada automaticamente por el compilador y sirve como area de spilling
de registros cuando se acaban. Cuenta con las mismas desventajas que la memoria global.

La GPGPU cuenta con multiples niveles de memorias cache para poder aminorar el hecho
de que el principal cuello de botella del computo es la latencia en los accesos a memoria
global. Estas se dividen en tres:

\begin{itemize}
  \item Cache L1
  \item Cache L2
  \item Cache de textura
\end{itemize}

La cache L1 es dedicada por SM. Esta cach\'e fue introducida en Fermi y su dise\~no hace que
tambien esta dedicada a la memoria compartida, por lo que es posible en tiempo de ejecuci\'on
darle directivas a la GPGPU que asigne m\'as memoria cache o m\'as memoria compartida,
permitiendo a los bloques tener mayores espacios de memorias compartidas o mayores hit rate de cach\'es.

La cache L2 es com\'un a todos los SM de la GPGPU, donde, a partir de Fermi en NVIDIA, todos
los accesos de lectura y escritura a memoria global y textura pasan a traves de esta. ~\cite{NvidiaFermi}

La cache de textura es una cache que presenta no solo localidad espacial, como la mayoria
de las caches de procesadores normales (i.e. dato-1, dato, dato+1, etc.), sino que se le
puede agregar el concepto de dimensiones, para poder modelar datos en mas de una dimension.
Esto la convierte en una herramienta clave a la hora de minimizar los accesos a matrices
no solo por filas sino por columnas. Esta memoria se debe definir en momento de compilacion,
en el codigo y no es automatica, porque esta cach\'e tiene limites espaciales (necesarios
para poder definir areas de memoria sobre la cual operar) y a su vez se debe acceder
a los dayos subyacentes a traves de funciones especificas. Una caracteristica adicional
de esta cach\'e es que como necesita resolver estos accesos extra\~nos a la memoria, cuenta
con una unidad propia de resoluci\'on. Esta unidad tiene limitantes a cuanto podemos
exigirle, ya que no posee un ancho de banda suficiente como para resolver todos los
accesos a memoria globales podria servir, asi que hay que usarla judiciosamente.

Para soportar una arquitectura masivamente paralela, se debe usar una ISA
(\texttt{Instruction Set Architecture}) dise\~nada especialmente para el problema. Esta ISA
debe poder soportar conceptos fundamentales del computo GPGPU: grandes cantidades de registros,
operaciones en punto flotante de precision simple y doble y fused multiply-add. Ademas,
el c\'odigo compilado para GPGPU debe ser agn\'ostico al dispositivo que lo va a correr, por
lo cual la paralelizaci\'on no debe estar demasiado atada a este, sino que el dispatching
lo debe poder determinar la placa en tiempo de ejecuci\'on. De este modo, la paralelizaci\'on
deberia ser la m\'as apropiada en cada dispositivo, sin importar la generaci\'on que sea. Un \'ultimo
requerimiento clave de esta ISA es que debe poder soportar un m\'inimo de \texttt{tuneo} manual,
para poder construir partes claves de ciertas librerias frecuentemente usadas (como las BLAS)~\cite{NvidiaFermi}.


\subsection{Requerimientos de un problema para GPGPU}
En un principio, al ser GPGPU una arquitectura con completo poder expresivo, se puede
calcular cualquier cosa. Sin embargo, hay altos costos en los que se incurren antes de
poder ejecutar fragmentos de codigo. Un problema debe exhibir las siguientes caracteristicas
para que valga la pena poder pensar en correrlo para GPGPU.
\begin{enumerate}
  \item \label{req:paralelo} El problema debe tener una gran parte paralelizable.
  \item \label{req:float} El problema debe consistir mayormente de operaciones numericas de punto flotante.
  \item \label{req:matrix} El problema debe poder ser modelado mayormente en arreglos o matrices.
  \item \label{req:transf} El tiempo de computo debe ser muy superior al tiempo de transferencia de datos.
\end{enumerate}

Item \ref{req:paralelo} se refiere a que debe existir alguna forma de partir el problema
en subproblemas que puedan realizarse simultaneamente, sin que haya dependencias de
resultados entre si. Si el problema requere partes seriales, lo ideal es que se las
pueda concebir las partes paralelas sean etapas de un pipeline de procesos, donde
cada una de estas exhiba caracteristicas fuertemente paralelas. Como las arquitecturas
masivamente paralelas tienen como desventaja una menor eficiencia por core, si el
problema no se puede dividir para maximizar la ocupancia de todos los cores, no
va a ser posible superar en eficiencia a los procesadores seriales.

Item \ref{req:float} habla de que el m\'etodo de resoluci\'on de los problemas debe
consistir del uso de metodos numericos. El set de instrucciones de las arquitecturas
de GPGPU estan fuertemente influenciados por las aplicaciones 3D que las impulsaron
en un principio. Estas consisten mayormente de transformaciones de algebra lineal
para modelar luces, hacer renders o mover puntos de vistas. Todos estos problemas
son inherentemente de punto flotante, por lo cual el set de instrucciones, las ALUs
internas y los registros tienen optimizado para este caso de uso. Las operaciones
en numeros enteros no son el fuerte de esta arquitectura y suelen ser realizados
mas eficientemente por procesadores de proposito general.

Item \ref{req:matrix} menciona que los problemas que mejor se pueden tratar en esta
arquitectura se pueden representar como operaciones entre arreglos o matrices de
dos, tres o cuatro dimensiones. Las estructuras de datos que no estan secuenciales
en memoria pueden incurrir en multiples accesos a esta para recorrerlas, y en las
arquitecturas GPGPU son estos los que generan el mayor cuello de botella. Ademas,
son dificiles de paralelizar en multiples subproblemas. Tener como parametros de
entrada matrices o arreglos que se puedan partir facilmente incurre en minimos
overheads de computo y permite aprovechar mejor las caches y las herramientas de
prefetching que brinda la arquitectura.

Item \ref{req:transf} ataca uno de los puntos criticos de esta arquitectura. Para poder
operar con datos, se requiere que estos esten en la memoria de la placa, no la memoria
de proposito general de la computadora. Esto quiere decir, que se deben hacer copias
explicitas entre las memorias, ya que ambas tienen espacios de direcciones independentientes.
Esta copia se realiza a traves de buses que, a pesar de tener un enorme throughput de
datos, tambien tienen una gran latencia (orden de milisegundos). Por lo tanto, para minimizar
el tiempo de ejecuci\'on de un programa usando las GPGPU, se debe considerar tambien el
tiempo de transferencia de datos a la hora de determinar si el beneficio de computar en
menor tiempo lo justifica. Las nuevas versiones de CUDA buscan brindar nuevas herramientas
para simplificar este requerimiento, proveyendo espacio de direccionamiento unico \ref{CUDA4} y
memoria unificada \ref{CUDA6}, pero siguen siendo copias de memoria a traves de los
buses (aunque asincronicas).

Estas caracteristicas limitan enormemente la clase de problemas que una GPGPU puede
afrontar, y suelen ser una buena heuristica para determinar de antemano si vale la pena
invertir el tiempo necesario de la implementaci\'ony ajuste fino.

\subsection{Funcionamiento de un GPGPU}

El c\'odigo escrito para CUDA se puede compilar a dos targets distintos; uno es
el codigo binario nativo de la placa target donde se va a correr y el otro es un
codigo intermedio, llamado c\'odigo PTX, que se JIT compila por el driver CUDA
antes de enviar a la placa, de modo que sea portable entre placas y arquitecturas
(retrocompatibles).

Cuando se lanza un kernel de ejecuci\'on, este c\'odigo binario se carga en la DRAM propia
de la placa. Este codigo va a ser ejecutado por todos los bloques que se hayan definido cuando
se lanz\'o el kernel. El ``Thread Engine'' del GPGPU se encarga de repartir los bloques a los
SM (Streaming Multiprocessors). Cada SM luego va a ejecutar de a grupos de 32 threads, cada uno
de ellos corriendo sobre un SP (Streaming Processor). A traves de sus 2 o 4 unidades de dispatch de warp
(de acuerdo a la generaci\'on de los chips), el SM puede dinamicamente cambiar el warp que se esta ejecutando en sus SP,
escondiendo la latencia de los stalls de pipelines forzosos para la ejecucion de instrucciones de multiples clocks.
Estos threads a su vez obtienen sus registros de un register file comun a todos los SP. Un scoreboard
es mantenido por cada dispatcher de warps para poder determinar cuales estan listos para correr. En Kepler
este scoreboard es simplificado ya que las latencias de las operaciones matem\'aticas es conocido, por
lo cual se puede reemplazar por contadores m\'as sencillos. ~\cite{NvidiaKepler}

Todos los SP de un GPGPU son sim\'etricos y cuentan todos con las mismas unidades de FP y de Enteros,
y entre varios cores comparten una o dos unidades de precisi\'on doble y una unidad de funciones especiales.
En Fermi no se pueden ejecutar simultaneamente instrucciones de precision doble y simple, y como
las de doble requerien ambos pipelines, suelen disminuir considerablemente el aprovechamiento
de los cores. En Kepler, al usar 4 unidades de despacho de warp, puede elegir una mejor
combinaci\'on para poder ejecutar simultaneamente instrucciones que no dependan de las mismas
unidades funcionales en el pipeline. ~\cite{NvidiaKepler}

Como estos procesadores implementan el estandar IEEE754-2008, cuentan con precision simple y
doble correcta, por lo cual dentro de las operaciones nativas, cuentan con instrucciones FMA
(Fused multiply-add) que no pierden precisci\'on en etapas intermedias de redondeo.

%Como el c\'odigo se ejecuta de manera sincr\'onica entre todos los threads del warp, las instrucciones
%condicionales proveen un problema para esta arquitectura. Como las distintas ramas del condicional
%son instrucciones excluyentes, no las deben ejecutar todos los threads simultaneamentes. El procesador
%deshabilitaba los cores que manejaban los threads de las ramas que no se ejecutaban del
%condicional~\cite{NvidiaTesla}. Para disminuir la cantidad de instrucciones de condicionales
%que se ejecutan, se cuenta con ~\cite{NvidiaFermi} predicaci\'on en todas las instrucciones de la ISA.

\subsection{Diferencia entre CPU y GPU - Procesadores especulativos}
Viendo la historia de CPU, podemos hacer un recorrido de los problemas que los dise\~nadores de procesadores
fueron enfrentandose a lo largo del tiempo mirando los componentes que fueron apareciendo.

(insertar foto de die de nehalem)

Caches - Pipelines - Predictores - M\'as cache - M\'as controladores de memoria - Multiple issues

Lo importante ac\'a es observar el patron: ``no desechar algo que podramos necesitar pronto'',
``intentar predecir el futuro de los condicionales'', ``intentar correr multiples instrucciones a la vez
porque puede llegar a bloquear en alguna de ellas.''

Todos estos problemas convierten al CPU en una maquina que gira alrededor de la especulacion,
de los valores futuros que van a tener las ejecuciones, del probable reuso de datos.
Podemos observar entonces que las unidades que verdaderamente hacen la ejecuci\'on de las operaciones,
las ALU, son muy pocas en comparaci\'oncon la cantidad de dispositivos de soporte que estan
sobre el die del CPU.

En contraste, las arquitecturas GPGPU son verdaderas maquinas de computo masivo. Estan dise\~nadas para
resolver una y otra vez operaciones muy bien definidas (intrucciones de punto flotante en su mayoria).
Comparativamente con un CPU, las ALU de las GPU son bastante pobres y lentas. No funcionan a las mismas
velocidades de clock y deben estar sincronizadas entre si. Pero la gran ventaja esta en la cantidad.
Un CPU cuenta con unas pocas ALU por core, dependiendo de su algoritmo de scheduling interno
(alrededor de 16 cores por die de x86 es el tope de linea ofrecido actualmente). Un GPU cuenta con miles de ALUs en total
(m\'as de 2500 CUDA Cores en Tesla K20).
En contraste, un GPU dispone de pocas unidades de soporte del procesamiento. No tiene pipelines especulativos, el tama\~no de las caches
estan a ordenes de magnitud de las de CPU, la latencia a las memorias principales de la GPU estan a
decenas de clocks de distancia, etc. La arquitectura asume que siempre va a tener m\'as trabajo
para hacer, por lo cual en vez de intentar solucionar los pitfalls de un grupo de threads, directamente
los reschedulea para m\'as adelante y continua procesando otro warp de threads. Basicamente durante del
disen\~no de la arquitectura GPGPU buscaron resolver el problema del computo masivo pensando en hacer
m\'as cuentas a la vez y recalcular datos viejos de ser necesario. Esto es una marcada diferencia contra
los CPU, que estan pensado en rehacer el menor trabajo posible y intentar mantener todos los datos que pueda en
las memorias caches gigantescas.

Como los CPU tienen que poder soportar cualquier aplicacion, no pueden avocarse de lleno a una sola
problematica. El hecho de no tener que dise\~nar un procesador de proposito general permitio un cambio radical
a la hora de pensar como concebir una arquitectura de gran throughput auxiliar al procesador, no reemplazandolo
sino mas bien adicionando poder de computo.~\cite{GlaskowskyFermi}

Las arquitecturas Fermi (y agregar AMD) han concebido el dise\~no de un procesador de alto desempe\~no.
Su meta principal es poder soportar grandes cantidades de paralelismo, mediante el uso de procesadores
simetricos, pero tomando la fuerte restricci\'on de ``no siempre tiene que andar bien''. Es decir, ellos
mismos asumen que el codigo que van a correr esta bien adaptado a la arquitectura y no disponen
casi de mecanismos para dar optimizaciones post-compilacion. Relajar esta restricci\'on
les permitio romper con el modelo de computo de CPU y definir nuevas estrategias de paralelismo,
que no siempre se adaptan bien a todos los problemas, pero para el subconjunto de los desafios que se
presentan en el area de HPC y de videojuegos han probado ser un cambio paradigm\'atico.

\subsection{Idoneidad para el problema}
El problema de QM/MM enfrentado en este trabajo cuenta con mutiples operaciones matematicas de gran
volumen de calculos. En particular, las operaciones matriciales son los cuellos de botella en este
problema.
Para obtener los valores numericos de densidad buscados en los puntos, se deben obtener las derivadas primeras
y segundas, lo cual implica hacer multiples operaciones de multiplicaci\'onmatricial. Estas multiplicaciones
pueden ser paralelizadas por columnas y por filas. Estos problemas estan fuertemente estudiados como paralelizarlos
en la literatura, ya que los problemas de algebra lineal los requieren constantemente.
En nuestro caso, se requieren para un sistema cientas de estas multiplicaciones, algunas con matrices de mas de
$500^2$ elementos. Al consistir este proyecto un sistema de resoluci\'on num\'erico de las f\'ormulas de QM/MM,
los problemas enfrentados eran casi en su totalidad de operaciones de punto flotante. Luego, dados las
caracteristicas de contar con un fuerte nivel de paralelismo en los cuellos de botella y de ser operaciones
mayormente de punto flotante, determinamos que el uso de las GPGPU era idoneo, en comparaci\'on con arquitecturas
de proposito general con menos GFLOPS.

\subsection{Arquitectura y features Xeon Phi}

\subsubsection{Hardware}

La arquitectura del Xeon Phi, parte de la linea de Intel Many Core Architecture (MIC), est\'a esquematizada
en la figura~\ref{fig:xeon_phi_arch}. Cada procesador esta esquematizado en~\ref{fig:xeon_phi_core}

La base de esta arquitectura consiste entre 60 y 80 cores SMP (\textit{Symmetric Multiprocessing}), con lo cual todos
ellos comparten la misma memoria principal. Cada uno de los cores tiene un \textit{clock rate} de 1 GHz,
con una arquitectura similar al set de instrucciones de Intel IA-32. Las principales diferencias con este
conjunto de instrucciones son el soporte para direccionamiento a 64 bits y nuevas instrucciones de vectorizaci\'on.

Cada procesador tiene adem\'as soporte para 4 \textit{threads}, es decir, 4 hilos de ejecuci\'on diferentes. Adicionalmente,
cada \textit{thread} puede ejecutar dos instrucciones por ciclo de clock, mediante el uso de dos pipes: \textit{V-pipe} y \textit{U-pipe}.
Algunas instrucciones, sin embargo, solo puede ser ejecutadas en una de las dos: por ejemplo las instrucciones de vectorizaci\'on solo pueden
ser ejecutadas en la \textit{U-pipe}. Para ejecutar estas instrucciones se dispone de una unidad de vectorizaci\'on (VPU, \textit{Vector Processing Unit}).
Esta unidad cuenta 32 registros SIMD (Single Instruction Multiple Data) de 512 bits. La latencia de estas instrucciones es de 4 ciclos de clock pero permiten
operar sobre 16 valores de punto flotante de precisi\'on simple (\textit{float}) a la vez.

Adem\'as de la unidad de vectorizaci\'on y la unidad escalar, cada procesador cuenta con 32 Kb de cache L1 y 512 Kb de cache
L2. Estas caches son asociativas \textit{8-way} y su linea de cache tiene 64 bytes. La coherencia de cach\'e se mantiene
mediante un directorio distribuido de \textit{tags} (v\'ease figura~\ref{fig:xeon_phi_arch}) dividido en 64 secciones e implementado
por hardware. La memoria principal consiste de memoria RAM GDDR5 en la placa, de 8 GB con velocidad de transferencia de 5.5 GT/s en 16 canales y con transferencia de 4
bytes. Esto nos da un l\'imite de ancho de banda te\'orico de 352 GB/s pero detalles de implementaci\'on de los chips limitan este valor a 200 GB/s.

Cada procesador, puede realizar pedidos de memoria independientes sin que la memoria se convierta en un cuello
de botella importante~\cite{Fang}. Sin embargo, los 4 \textit{threads} dentro de un \textit{core} ven sus accesos a memoria serializados.

Adicionalmente los cores est\'an conectados por dos anillos bidireccionales que les permite comunicarse entre si. La velocidad de
comunicaci\'on es suficiente para considerar que todos los procesadores son sim\'etricos (es decir, cada procesador puede comunicarse con
cualquier otro con un \textit{overhead} despreciable)~\cite{Fang}.

Por \'ultimo, cada procesador tiene un \textit{in-order pipeline} de corta longitud, diferencia importante con los cores de un procesador
estandar de la arquitectura x86. El \textit{pipeline} corto implica que las operaciones escalares no tienen latencia y las vectoriales tienen baja latencia,
y el costo por \textit{branch misprediction} es bajo~\cite{IntelXeonWhitePaper}. Este punto diferencia fuertemente al Xeon Phi de aceleradores de computo como las GPGPU, 
que tienen alto costo en las bifuraciones de decisiones. Sin embargo, el Xeon Phi no ejecuta instrucciones de manera \textit{out-of-order}, lo cual implica que muchas 
t\'ecnicas de optimizaci\'on que explotan el paralelismo a nivel instrucci\'on usuales en arquitecturas como x86-64 no son aplicables.

La otra diferencia es nuevas instrucciones de vectorizaci\'on, incompatibles con sets de vectorizaci\'on anteriores de Intel (por ejemplo AVX o SSE 4.1).
Estas operaciones incluyen implementaciones por \textit{hardware} de operaciones com\'unes en HPC: rec\'iproco de un valor, ra\'iz cuadrada, potencia y
exponenciaci\'on, y operaciones m\'as relacionadas con la memoria como por ejemplo \textit{scatter and gatter} y stores \textit{streameados} de manera de aprovechar
mejor el ancho de banda que tiene la arquitectura.

El Xeon Phi es un coprocesador, lo cual implica que necesita ser instalado sobre una computadora que sirva de \textit{host}. La comunicaci\'on con este host
ocurre a trav\'es de un bus PCI Express, no comparten ni memoria ni otros perif\'ericos como por ejemplo disco duro. Existen dos m\'etodos para acceder al Xeon~\cite{BookXeonPhi}

\begin{enumerate}
    \item Nativo: El Xeon Phi permite correr c\'odigo directamente, mediante el uso de SSH (Secure SHell). Esto es gracias a la presencia de BusyBox Linux como sistema operativo,
    lo cual da soporte de sistema de archivos y entorno de ejecuci\'on. La interfaz utiliza TCP/IP virtualizado mediante el bus PCI Express. Si bien el coprocesador no tiene acceso a \textit{storage} persistente (puesto que el sistema operativo esta montado sobre la memoria) esto puede resolverse utilizando un sistema de archivos remoto montado en la memoria del host.
    \item Offloading: El \textit{host} puede delegar la ejecuci\'on de ciertas porciones de c\'odigo al coprocesador. Esto requiere que los datos necesarios para el c\'omputo sean copiados del \textit{host} al Xeon Phi, lo cual puede implicar que el bus puede ser un cuello de botella importante (puesto que los datos de entrada, y la salida deben ser movido al Xeon Phi y traidos de vuelta al finalizar el c\'omputo).
    \item Simétrico: En este modo de ejecuci\'on se piensa al Xeon Phi y su host como dos nodos en un \textit{cluster} de c\'omputo, y al bus PCIe como una red de alta velocidad.
Este modo es especialmente interesante si se dispone de m\'as de un Xeon Phi en un mismo host, y se utiliza una interfaz de pasado de mensajes entre ellos como por ejemplo MPI (\textit{Message Passing Interface}).
\end{enumerate}

Por \'ultimo, en pos de simplificar el trabajo de adaptar una aplicaci\'on a usar el Xeon Phi, el mismo provee una unidad de monitoreo de \textit{performance} (\textit{Performance
Monitoring Tool}, PMU). Esta unidad permite la colecci\'on de informaci\'on del coprocesador, aunque no tiene soporte para ciertas caracter\'isticas com\'unes a los procesadores
de lineas m\'as est\'andar de Intel (por ejemplo: \textit{sampling} preciso de eventos por \textit{hardware}, como por ejemplo ciclos por instrucci\'on o \textit{misses} de cache).

