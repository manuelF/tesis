\begin{center}
\large \textsc{RESUMEN}
\end{center}
\vspace{1cm}

\noindent

El modelado computacional es una herramienta indispensable para poder entender
fenomenos que no se pueden observar salvo en casos elementales. Todas las disciplinas
f\'isicas presentan casos de estudios que no se pueden observar empiricamente, ya
sea que son eventos inviables de recrear, o bien no existen a\'un los metodos para
observarlos. La quimica es una de estas ciencias, donde la computacion ha hecho mella
en el estudio del comportamiento de particulas subat\'omicas, imposibles de ver usando
incluso las t\'ecnicas mas avanzadas. En los ultimos a\~nos, han surgido nuevas posibilidades
de modelado de mecanica cuantica, empleando las teorias de la densidad del funcional de Kohn-Sham
y los metodos de Hartree-Fock. La aplicaci\'on de estas ideas junto con las teorias convencionales
de mecanica clasica para estudiar el comportamiento de sistemas biologicos han sido meritorios
del premio Nobel de quimica del a\~no 2013.

Realizar los c\'alculos necesarios para simular un modelo no trivial usando estos m\'etodos
no hubiera sido posible sin haber aprovechado el crecimiento exponencial del poder de
computo de estas ultimas d\'ecadas. En los ultimos a\~nos ha aparecido una nueva gama de
procesadores, oriundos del area de juegos de video, que ofrecen una enorme capacidad de
computo asequible, aparejados a un cambio radical de como pensar la resoluci\'on de
todos estos sistemas num\'ericos. Describiremos a fondo estas arquitecturas modernas,
y estudiaremos como poder acelerar aun m\'as una aplicacion existente de QM basada en
la teoria de densidad funcional. Se espera que estos cambios permitan realizar experimentos
a escalas novedosas, ya sea en complejidad de sistemas, en la cantidad de tiempo simulado o
bien en llevar esta herramienta de QM a nuevas areas de quimica computacional aun sin explorar.


\bigskip

\noindent\textbf{Palabras claves:} QM/MM, DFT, Xeon Phi, CUDA, \textit{scheduling}.
