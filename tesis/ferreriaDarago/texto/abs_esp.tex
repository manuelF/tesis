\begin{center}
\large \textsc{RESUMEN}
\end{center}
\vspace{1cm}

\noindent

En este trabajo se busca analizar distintas arquitecturas de procesadores especializados
en c\'omputos masivos. Para estudiarlos, se trabaj\'o sobre una implementaci\'on existente
de una aplicaci\'on de qu\'imica cu\'antica que resuelve c\'alculos de estructura electr\'onica.
Originalmente, se contaba con implementaciones de los algoritmos de la Teor\'ia de los
Funcionales de la Densidad (DFT) hechas en CPU y GPGPU, pero su performance no escalaba con los
dispositivos modernos, dados los cambios tecnol\'ogicos sucedidos desde su implementaci\'on original.
En este trabajo se busc\'o
aumentar la \textit{performance} de las partes que m\'as tiempo insum\'ian en el c\'alculo utilizando
las caracter\'isticas \'unicas de las arquitecturas de CPU y GPGPU modernas. La mejora de
rendimiento en los c\'alculos obtenida en la generaci\'on actual de GPGPU es del orden de ocho
veces, mientras que la de CPU pudo escalar m\'as de veinte veces.

Se aplicaron en este trabajo tambi\'en m\'etodos para particionar tareas en CPU y GPU y como
balancear cargas en tiempo de ejecuci\'on, utilizando m\'ultiples GPU sim\'etricas en un mismo nodo
obteniendo resultados cercanos al optimo te\'orico.

Por otra parte, se pudo explorar una arquitectura nueva de c\'omputo masivo, la
tecnolog\'ia Xeon Phi de Intel para poder estudiar su aplicabilidad en aplicaciones reales, con
resultados poco alentadores, consistentes con otros estudios.

Finalmente, los resultados obtenidos reflejan la importancia de trabajar con el \textit{hardware}
apropiadamente, explotando las capacidades \'unicas de cada herramienta. Se deja asentada evidencia
de como se puede realizar particiones h\'ibridas GPU y CPU, con potenciales grandes mejoras
de performance y gran escalabilidad usando los recursos disponibles. Se espera que el an\'alisis
realizado y las t\'ecnicas implementadas durante este trabajo sirvan tambi\'en para guiar mejoras
de \textit{performance} futuras de aplicaciones similares para aprovechar los crecientes recursos de c\'omputo
disponibles.

\bigskip

\noindent\textbf{Palabras claves:} QM/MM, DFT, Xeon Phi, CUDA, HPC, \textit{scheduling}.
