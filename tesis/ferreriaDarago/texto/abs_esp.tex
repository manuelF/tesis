\begin{center}
\large \textsc{RESUMEN}
\end{center}
\vspace{1cm}

\noindent

En este trabajo se busca analizar distintas arquitecturas de computadoras especializadas
en computos masivos. Para estudiarlas, se trabaj\'o sobre una implementacion existente
de una aplicacion de quimica cuantica que resuelve sistemas de estructura electronica.
Originalmente, se contaba con implementaciones de los algoritmos de la Teoria de los
Funcionales de la Densidad (DFT) hechas en CPU y GPU, pero su performance no escalaba con los
dispositivos modernos, dados los cambios tecnlogicos sucedidos desde entonces. Se busc\'o
aumentar la performance de las partes que mas tiempo insumian en el calculo utilizando
las caracteristicas unicas de las arquitecturas de GPU y CPU modernas. La mejora de
rendimiento en los calculos obtenida en la generacion actual de GPU es del orden de ocho
veces, mientras que la de CPU pudo escalar mas de veinticuatro veces.

Se aplicaron en este trabajo tambien m\'etodos para particionar tareas en CPU y GPU y como
balancear cargas en tiempo de ejecucion utilizando multiples GPU simetricas y asimetricas,
obteniendo resultados.

Por otra parte, se pudo explorar una arquitectura nueva de computo masivo, la
tecnologia de Xeon Phi de Intel para poder estudiar su aplicabilidad en aplicaciones reales, con
resultados prometedores a futuro.

Finalmente, los resultados obtenidos reflejan la importancia de trabajar con el hardware
apropiadamente, explotando las capacidades unicas de cada tecnologia. Se deja asentada evidencia
de como se puede realizar particiones hibridas GPU y CPU, con potenciales grandes mejoras
de performance y gran escalabilidad usando los recursos disponibles. Se espera que el analisis
realizado y las tecnicas implementadas durante este trabajo sirvan tambien para guiar mejoras
de performance de aplicaciones similares para aprovechar los crecientes recursos de c\'omputo
disponibles.

\bigskip

\noindent\textbf{Palabras claves:} QM/MM, DFT, Xeon Phi, CUDA, \textit{scheduling}.
