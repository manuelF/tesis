\subsection{Estructura inicial del programa}

El programa originalmente estaba concebido para ser corrido en GPU nVidia GTX8800.
Luego, las cosas que se usaron para este desarrollo consistieron en la libreria CUDA version
2, para arquitecturas a lo sumo SM20.

La estructura del programa era la siguiente:
\begin{enumerate}
\item Se determinan los mallados del sistema
\item Se clasifican en cubos y esferas
\item Para cada elemento, se resuelve el sistema
\item Esto se repite hasta que converga a lo sumo en una cantidad limitada de pasos, o diverga
\end{enumerate}

La resoluci\'on del sistema en si esta compuesta de varios pasos:
\begin{enumerate}
\item Se obtiene el valor de la funci\'on de onda en cada punto de la malla
\item Se generan las matrices con los gradientes y los hessianos de cada funci\'on
\item Se calculan las densidades y el producto matricial por bloques por funci\'on
\item Se calcula la matriz de coeficientes de las fuerzas entre las particulas
\end{enumerate}


\subsection{Cuellos de botellas originales, limitantes estructurales}

El cuello de botella principal que presentaba este c\'odigo era en la
resoluci\'on del sistema, especificamente en el c\'odigo que calculaba la matriz de Kohn-Sham.
En el GPU, esta funci\'on insumia el 95\% del tiempo total, que en sistemas de gran cantidad
de puntos, estaba en el orden de minutos. Los problemas principales que mostraba esta funci\'on fueron:
\begin{itemize}
\item Cantidad de accesos a memoria global
\item Falta de accesos coalescentes
\item Sobreuso de la memoria compartida
\item Subsaturacion de los SM
\end{itemize}

\subsection{Accesos a memoria global}

\subsection{Falta de accesos coalescentes}
Asi como a los CPU les interesa realizar accesos alineados a memoria, a los GPU les
interesa aun mas. El termino coalescencia de memoria se aplica a GPU mediante
la organizacion de los accesos a memoria de manera ordenada y predecible. La l\'ogica
detras de esto es que cuando el GPU accede a memoria de manera alineada, puede traer
entre 16 y 64 bytes en una sola lectura, mientras que si debe acceder de manera no
alineada a memoria, o con threads que no acceden de manera predecible a esta, entonces
se deberan serializar los accesos y separar en m\'ultiples accesos a memoria global para leer
esta informaci\'on. Este problema solamente se agrava si se debe hacer muy frecuentemente
por cientos de threads, como es el caso de los bloques con gran nivel de paralelismo explicito.

\subsection{Subsaturacion de los SM}
La subsaturacion de los SM se da en los casos donde haya SM que esten listos para correr
c\'odigo pero que no puedan hacerlo porque tienen contencion en algunos de sus recursos.
La m\'etrica usada para determinar esta saturaci\'on es la ocupancia de los SP.
Esta es la proporci\'on de threads activos sobre el total de threads disponibles de un bloque.

Existen en esta arquitectura principalmentre tres recursos que, en un principio, parecen
ilimitados pero en realidad son finitos y compartidos por los procesadores de la GPGPU.
Estos son:
\begin{itemize}
\item Cantidad total de threads por bloque.
\item Cantidad total de registros usados por thread.
\item Cantidad de memoria compartida por bloque.
\end{itemize}

El mecanismo de scheduling de los SM funciona asignando un bloque a cada SM, que
va a correr sin preemption hasta que terminen todos sus threads asignados. Idealmente, cada
bloque cuenta con una cantidad de threads suficiente para poder esconder la latencia
de las ejecuciones mediante un cambio de contexto. La arquitectura GPGPU esta dise\~nada
para este fin, por lo cual se cuenta con un mecanismo de cambio de contexto de costo cero~\cite{NvidiaFermi} para
poder empezar a correr los threads de un warp diferente, del mismo bloque.
Si el bloque no cuenta con suficiente cantidad de threads para poner a correr de manera
concurrente, el SM va a forzosamente esperar que finalicen las operaciones de alta latencia
de estos warps sin nada que hacer mientras tanto. Si, por el contrario, se contasen con
miles de threads por bloque, entonces es posible que las operaciones que sirvan
para sincronizar los threads de todo un bloque en un punto espec\'ifico antes de proseguir
(un barrier) sean excesivamente costosas.

La arquitectura GPGPU de NVIDIA organiza los registros de todos los threads en un unico
register file, com\'un a todos los bloques. Como cada thread usa decenas de registros para guardar
los computos intermedios, NVIDIA decidio unificarlos, ya que es muy variable la cantidad que va a usar
cada kernel de ejecuci\'on. Una de las grandes diferencias entre Fermi y Kepler es la cantidad m\'axima de
registros por thread. Mientras que Fermi permitia hasta 63 registros, Kepler permite hasta 255. Esto
es positivo para poder correr bloques de pocos threads pero gran cantidad de registros. Por otro lado,
aumenta la presion sobre el register file. Cuando se lanzan muchos threads
que puedan estar corriendo paralelamente entre todos los SM de la GPU, es posible que se supere
la cantidad m\'axima de registros presentes en el register file. Esto fuerza a que el scheduler
no pueda poner a ejecutar m\'as bloques que los que pueda soportar este recurso, dejando SP ociosos.

Finalmente, al igual que con los registros, la memoria compartida es un recurso limitado. Como
solamente se cuenta con hasta 48Kb (Fermi-Kepler) de memoria de este tipo para ser repartida entre
todos los bloques que esten corriendo en todos los SM, el scheduler debera decidir no poner a ejecutar
m\'as bloques simultanemante que los que pueda soportar la cantidad de memoria compartida.

El problema tratado dentro de esta tesis cont\'o con todos estos limitantes. Afortunadamente,
las herramientas de profiling usadas remarca estos limitantes constamente, haciendolas
fundamentales a la hora de evaluar como proseguir en la busqueda de optimizaciones de c\'odigo.

\subsection{Cambios en el threading}
%Cambiamos de blocks por puntos, a blocks por funci\'on.
De los 3 kernels que originalmente componian la iteracion de SCF, el que se encargaba
del calculo de la densidad insumia el 94\% del tiempo total de uso de la GPU. Minimizar
el tiempo de ejecuci\'on de esta funcion resultaba vital para poder disminuir el tiempo
de convergencia de SCF.

El cuello de botella fundamental radicaba en como se distribuia el trabajo de computo
entre los kernels. La estrategia de paralelizaci\'on original determinaba la partici\'on
del sistema a resolver instanciando un bloque por cada punto (\texttt{blockId.x $\in$ group\_m}),
una cantidad fija de threads (usando \texttt{threadId.x $\in$ BLOCK\_SIZE})
Los threads servian para reutilizar la memoria compartida; cada uno hacia la lectura
de un $F_i$ que se lo compartian entre todos, y luego todos usaban un $F_j$ distinto, para
hacer la multiplicaci\'on de filas por columnas.

Esta distribuci\'on resultaba natural al problema, pero visto con mayor detalle, esto
implicaba una cantidad de cuentas innecesarias que podian ser eliminadas.

Uno de los cambios importantes fue crear mas bloques, para las particiones que tienen
mas funciones. Decidimos usar otra dimensi\'on de los bloques (\texttt{blockId.y}),
para determinar cuantos grupos de threads van a hacer falta para procesar completamente
todas las funciones de ese punto. Llamamos a este parametro $altura\_bloques$. Se
calcula para cada particion entonces como $altura\_bloques = {group\_m}/{BLOCK\_SIZE}$
Para todas las particiones chicas, este valor no supera a 1. En los cubos y esferas m\'as
grandes (de los sistemas probados), la altura puede ser hasta 6. Esto significa una gran
cantidad de bloques adicionales con respecto al m\'etodo anterior.
%%
%%

Despues de haber hecho el cambio de la parelelizacion, estudiamos realizar
mas de un punto por thread. Esto sirve para aprovechar un par de lecturas que son comunes
entre dos funciones. Con este cambio, se instancian menos bloques (la mitad de la dimension $y$
definida para esto) y se pueden ocultar algunas latencias de acceso, pero
cada grupo de threads lleva m\'as tiempo y usan m\'as registros. Esta estrategia es similar
a un loop unrolling manual, aplicado a la arquitectura GPU.

Un punto de intensa discusion durante estos cambios es el valor de \textit{BLOCK\_SIZE}.
Para nuestro problema, decidimos utilizar un numero de threads por bloque m\'ultiplo del
tama\~no de un warp (32 threads). Esto permite estudiar como afectan en el tiempo de
procesamiento contar con uno o m\'as warps por bloque. Una ventaja de usar bloques de
32 threads, es que el costo de la sincronizaci\'on es exactamente cero. No se precisa
sincronizar nada puesto que los threads trabajan en lock-step sincronizados por warp.
Un bloque chico ademas nos permite usar mas memoria compartida por thread, dado que hay una
cantidad fija de memoria por bloque (entre 32Kb y 64Kb). Cuando se cuentan con muchos m\'as
threads, se debe reducir este uso por thread de modo que todos puedan ejecutar concurrentemente.

Dicho esto, la literatura ~\cite{farberCuda} sugiere siempre que sea posible
usar bloques grandes y con threads lo m\'as independientes posibles. Una gran cantidad de threads
en un bloque permite tener muchos m\'as warps para schedulear de modo de esconder las latencias de
operaciones y de a accesos globales. Sin embargo, contar con muchos threads hace que las
sincronizaciones sean mucho m\'as costosas. Ademas, como cada SM no cuenta con preemption
de bloques, contar con muchos threads por bloque hace que los recursos se mantengan
por largos periodos.

Inicialmente, este tama\~no se habia fijado en 128 threads por bloque, 4 warps. Utilizando
mas memoria compartida en el esquema de paralelizaci\'on para disminuir accesos a memoria global,
este valor resulto demasiado elevado y disminuia la posibilidad de ocupar todos los SM en dispositivos
Fermi y Kepler. Con solamente 32 threads, se podia maximizar la ocupacion de los SM, pero habia
muchos m\'as bloques. Finalmente, luego de disminuir un poco el uso de memoria compartida por
bloque, usando algunas ideas descriptas a continuaacion, pudimos fijar este valor en 64 threads
por bloque. Mostramos que tener solo un warp es bueno, pero mucho mejor a\'un es tener dos warps, porque
esto permite, con costo cero, poner a correr el otro para ocultar la latencia sin agrandar
demasiado el costo de la sincronizaci\'on.


%%%%%%%%%%%%% fruta underneath
Una cosa que se nota que este mecanismo de paralelizacion cuenta con
la ventaja de que casi no se compartian los datos entre los distintos threads, lo cual,
a priori, deberia haber ocasionado que no hubiera mucho mejor forma de realizar el computo.

Originalmente el c\'odigo generaba un bloque por cada punto en el grupo de puntos
que teniamos que solucionar, con una sola dimension por thread.
Como este era el kernel que insumia el 94\% del tiempo de todo el proceso, decidimos
atacarlo de raiz, cambiando la paralelizaci\'on. A pesar de que el profiler de CUDA nos indicaba
que tenia una buena occupancy, decidimos cambiar de fondo el encare del problema.


Sin embargo, con un detalle m\'as fino, es posible notar que este mecanismo hacia que
muchos threads hicieran cuentas innecesarias, que no contribuian a la reducci\'on total.
Esto daba una alta ocupancia ficticia, que buscamos acercar a la real. Adem\'as, a pesar
de que se compartian muy poca informaci\'on por la memoria compartida, tambien habia muchos
puntos de sincronización a nivel bloque.




\subsection{Cambios en la reducci\'on}
%Hubo que agregar reducci\'on de suma a nivel punto porque ya no se comparten mas la info
La reorganizacion de la paralelizacion del kernel del calculo de la densidad creo la necesidad
de varios pasos de reducci\'on que antes se hacian solos.

El primer paso, al ya no haber un bloque por punto, habra que totalizar el c\'alculo de la
suma de todos los elementos de la columna que acabamos de procesar para obtenerlo. Para reducir,
vamos a reutilizar la memoria shared que empleamos en el c\'alculo de la energia anterior. Cada
thread va a poner el valor final del computo realizado en su posicion correspondiente en el
la memoria compartida. Esto luego se ejecuta como una reducci\'on en \'arbol, donde cada
thread suma el valor de la posici\'on $x$ con el valor en $2*x$, si fuera este valido. Esto
luego se repite por la mitad de los threads, hasta que solo el thread 0 lo ejecuta,
generando exactamente un valor por bloque, que lo va a terminar escribiendo en la memoria
global.

Esta t\'ecnica de reducci\'on es sumamente conocida para arquitecturas distribuidas, generando
la respuesta en $O(log_2(n))$ pasos. La literatura de CUDA~\cite{cudaReductions} sugiere tecnicas adicionales para
minimizar a\'un mas el tiempo empleado en esta reducci\'on, pero considerando que hay, a lo sumo
6 operaciones, no hay necesidad de mejorar esto mas.

Como va a haber $altura\_{bloque}$ cantidad de escrituras para cada punto, va a entonces
hacer falta guardar estas cuentas parciales en memoria global. Para eso definimos una matriz
por cada uno de los parametros que debemos acumular, con tama\~no identico a la cantidad de bloques
del kernel \texttt{compute\_energy}, para que cada uno de estos escriba unicamente en una posici\'on
univocamente identificada de este. Estas matrices luego son de tama\~no $O(\#_{puntos} * altura_{bloque})$,
menos de 1Mb en el caso m\'as grande.

El siguiente paso de la reducci\'on consiste en acumular los $altura\_{bloque}$ valores descriptos
recien en un solo por punto. Esto implica encontrar donde estan en las matrices temporales las
partes de las cuentas, agregarlas y calcular el potencial correcto. Esto genera finalmente los
coeficientes para calcular la actualizaci\'on de la matriz de Kohn-Sham y los factores para el
c\'alculo de la matriz de fuerzas.

Esto se refleja en el c\'odigo como una llamada a un nuevo kernel adicional, posterior a la
cuenta de la densidad y con m\'ultiples matrices temporales adicionales. Este kernel es sumamente
eficiente porque ya todo el trabajo pesado lo hizo el anterior. Solo se tiene que realizar
a lo sumo $altura\_{bloque}$ sumas y una llamada a la funci\'on que calcula el potencial y densidad,
un kernel corto de alta intensidad aritm\'etica que realiza solamente operaciones matem\'aticas.
Finalmente, la acumulaci\'on finaliza, generando un valor por cada punto, lo mismo que se producia
anteriormente pero utilizando mucho mejor los recursos del dispositivo.

\subsection{Cambios en los accesos globales}
La arquitectura de las placas de video estan pensadas entorno al poder de computo.
Las decisiones tomadas por los diseñadores de las GPGPU se concentran alrededor
de paralelismo a lo ancho, poniendo un gran enfasis en la cantidad de nucleos. Luego,
se dispone de menor cantidad de espacio disponible en el die para las memorias.

Esta decision implica que la amplia mayoria de la memoria de la GPGPU se encuentra
localizada externa al procesador.  No solo esta fisicamente mas lejos, sino que
adem\'as la latencia para accederla es altisima. Es decir, el paradigma de
programaci\'on de las GPGPU gira entorno a esconder la gran latencia de los accesos
a las memorias globales.

Una de las memorias intermedias entre entre el procesador y la memoria global es
la memoria de textura. La memoria de textura es un cach\'e sobre la memoria global,
que esta focalizado alrededor de los accesos a memoria en varias dimensiones.
Estas memorias reciben su nombre de su funci\'on principal, que es en el area de los
sistemas de video. Los mapas de textura suelen ser grandes matrices que definen
tanto los colores sobre las superficies de los poligonos como los relieves.
El detalle crucial de estas memorias es que un miss en estas cache, provoca
que se traigan datos no solo contiguos en memoria, como pasa en las caches de
CPU normalmente, sino que ademas se traigan los datos en posiciones logicas contiguas,
es decir, variando las distintas dimensiones de la matriz subyacente.

Las memorias de textura se ajustan bien a los problemas de GPGPU, porque se relacionan
intimamente con los los mecanismos de paralelismo de CUDA. Como los problemas se pueden
dividir en bloques con threads en $x$, $y$, $z$, entonces tiene mucho sentido pensar
que las estructuras de datos subyacentes se van a acceder usando indices multidimensionales.

En nuestro problema, la memoria de textura se presenta como una soluci\'on para
los accesos bidimensionales de la matriz de RMM para el grupo de puntos.
Como esta matriz debe ser multiplicada por todos los valores de las funci\'ones,
derivadas primeras y segundas, se va a acceder a toda la matriz de RMM mas de
una vez por cada thread. Adem\'as, como se va a usar toda la matriz, y esta suele
tener un tamaño intermedio (es muy grande para memoria constante), el problema
suele entrar casi completamente en la memoria de textura.
La lectura bidimensional en este caso, se ajusta muy bien a los accesos por filas
y por columnas a la matriz.

El uso de la memoria de texturas agrega un recurso mas que tenemos que tener en
cuenta a la hora de profilear el c\'odigo. Para administrar los accesos a
la memoria de textura, cada multiprocesador tiene multiples "Unidades de textura".
Cuando dependemos de sobremanera de la memoria de textura para esconder la latencia,
se presentan contenciones sobre el acceso a las unidades de textura. Esto es
uno de los motivos por los cuales el procesador stallea los bloques hasta poder
ejecutarlos, cuando se liberan un poco mas los recursos.

%http://www.realworldtech.com/gt200/10/   << detalles sobre texture cache invalidation

\subsection{Cambios en los pasajes de informaci\'on intrawarp}
%Los shuffles que no anduvieron salvo en function
La arquitectura SM35, junto con CUDA5, trajeron aparejadas una herramienta interesante
para el manejo interno de los pasajes de informaci\'on intra-warp durante la ejecuci\'on.
Las instrucciones de shuffle, como asi las denomina NVIDIA, son instrucciones que facilitan
el pasaje directo de un registro de un thread en un warp, a otro, en un solo ciclo de ejecuci\'on.
Estas instrucciones existen en diversas maneras, con distintos propositos. Principalmente se
utilizan para pasar de un thread al siguiente (modulo el warp size) un registro para poder seguir
operando. Otro uso que puede tener mucho interes proximamente son las instrucciones de votaci\'on,
donde se evalua un predicado para todos los threads, y se setea o limpia un bit en el resultado
de respuesta si se cumplio el predicado para ese thread. Con esta herramienta, no es necesario
acceder a memoria compartida para poder pasar minima informaci\'on dentro de cada warp.

Nuestro uso de las funci\'ones de shuffle consistio en intentar eliminar lo m\'as que podiamos
los accesos a la memoria compartida, una fuente de bloqueos porque, a pesar de que ya corre
todos los bloques concurrentemente, leer elementos de ahi toman 4 ciclos en vez de uno solo
como en las funci\'ones de shuffle.

Probamos pasar de a un elemento y de a uno o dos vectores de 3 elementos, para comparar
cuan notable era el impacto del acceso mas veloz.

Finalmente concluimos que era una opcion valida para el pasaje de los valores de la funci\'on,
pero que el overhead de uso para cosas como los hessianos de la funci\'on no justifica el uso.
Adem\'as, como estas funci\'ones de shuffle solo estan presentes en las ultimas placas Kepler,
consideramos que el aprovechamiento marginal de los recursos no era lo suficientemente meritorio
de romper compatibilidad con las placas de la generacion Fermi anterior.


\subsection{Cambios en el almacenamiento de matrices temporales}
Una de las principales limitaciones de las GPGPU es la cantidad fija de memoria. Esta no es
expandible dado que esta soldada a la placa. Esto era aun m\'as notable cuando las placas
contaban con menos de 1Gb de memoria (A\~nos 2007-2008).
Para problemas de calculo num\'erico, esto era un limitante muy serio; los problemas que
normalmente entraban en la memoria principal de un CPU, no entraban completos en las GPU.
La decisi\'on tomada por muchas aplicaciones de entonces es compensar esto calculando
datos intermedios y tirandolos al final; teniendo que ser recalculados en las proximas iteraciones.

Esta estrategia es claramente impractica en CPU, puesto que se cuenta con mucha mas memoria
de uso general. En GPU era necesario por el faltante de memoria, pero no es tan notorio como en CPU,
puesto que estas cuentas se pueden hacer bastante m\'as rapido en GPU.

Cuando la aplicacion original se concibio, no habia siquiera placas Tesla de GPGPU apuntadas a HPC, con
m\'as memoria disponible que los modelos de consumidores. Aprovechando estos recursos actuales,
desarrollamos un m\'etodos para poder almacenar estos resultados temporales, de manera dinamica
durante la ejecuci\'on de la aplicaci\'on, para poder aprovechar este recurso que originalmente
era limitante pero ya no.

Para poder determinar que cosas van a ser guardadas en mem\'oria y que no, se determin\'o una heuristica
que define el orden de las particiones a solucionar. Esta heuristica estima que tama\~no van a
tener las matrices temporales a almacenar y ordena las particiones de menor a mayor. Esto
esta basado en el criterio de que, si bien es proporcional el tiempo de computo de estas matrices
temporales a la cantidad de funciones por grupo y cantidad de puntos (lo que determina el tama\~no
de la partici\'on), la constante es elevada. Determinamos entonces que es m\'as conveniente
aprovechar la memoria de la placa que almacena muchas matrices temporales de particiones chicas
a que almacene solamente un par de las grandes.

Para controlar la administracion de memoria, se realiza la estimaci\'on si la placa cuenta
con la suficiente memoria libre para guardar los datos a calcular, y si puede, se guardan de manera
permanente (hasta que la partici\'on se mueva a otro dispositivo o la liberaci\'on de recursos al
finalizar el programa). Esto ademas es configurable de modo que una corrida pueda usar un porcentaje
de la memoria con la que cuenta la placa, para poder correr multiples procesos de simulaci\'on concurrentemente.

Esta sencilla mejora permite explotar el hecho de que las placas hayan aumentado dramaticamente su
capacidad de almacenamiento, un recurso que hasta recientemente venia siendo un limitante podemos
convertirlo en una aceleraci\'on notoria.


\subsection{Cambios en las memorias compartidas}
%Cambiar los vec\_type4 por 3 en los accesos a la shared es mucho mejor, no hace falta alinear ahi
Otro de los problemas existentes del c\'odigo que quisimos atacar fue el mejor empleo de las
memorias shared. Estas son un recurso finito y muy importante, ya que son un limitante de
la ocupancia de los multiprocesadores. Como pueden correr una cantidad de bloques que, a lo sumo,
no superen los 48Kb de memoria shared simultaneamente entre todos, es imprescindible minimizar el
alocaci\'on de la memoria shared de modo que no estemos subutilizando los SM.

Una cosa que probamos, con un grado de exito variable, fue disminuir el tama\~no de los vectores
donde almacenamos las derivadas direccionales. Como nuestro problema es en tres dimensiones,
y los vectores estaban configurados para tener 4 valores por cada derivada, probamos llevarlos a
3, para que se ajusten a su consumo real de memoria. Este acercamiento no consider\'o el porque
se hizo asi de esta manera originalmente. Tener 4 valores consecutivos en memoria fuerza
al compilador a alinearlos a 16/32 bytes (simple y doble precision).
Esto presenta grandes ventajas a la hora de hacer transferencias de memoria global en los accesos,
por lo cual decidimos dejarlo como estaban.

Sin embargo, este mismo criterio no aplica a las memorias shared de la GPGPU. Como los accesos
a estas memorias se realizan de a 4 bytes y no de a 16/32, entonces no tiene ninguna ventaja
en particular realizar el alineamiento; ya estan alineadas porque los elementos de cada punto
son flotantes de precisi\'on simple o doble (4 y 8 bytes respectivamente). Adicionalmente, como el
cuarto valor no tiene forma de marcarse como algo que no sea padding de alineaci\'on, todavia se
opera normalmente con el, por lo que eliminarlo ahorra una operaci\'on de c\'alculo. Mas a\'un,
se presentan una disminuci\'on del 25\% de los recursos de la memoria compartida por thread,
sin ninguna desventaja a la hora de acceder a estos. Principalmente esta mejora permite
aumentar la cantidad de bloques corriendo concurrentemente en los SM, al punto de eliminar
la limitaci\'on presente debido a las memorias compartidas.

\subsection{Cambios en los condicionales}
La arquitectura CUDA representa un m\'odelo de computo pensado en el procesamiento secuencial masivo
de datos de punto flotante. Esto es herencia de su legado de placa gr\'aficas, que era un stream
constante de datos. Al generalizar la arquitectura para que sea de proposito general, entonces
surge el problema de ejecuci\'on condicional. Como el resultado de la evaluaci\'on puede ser distinto
para cada thread, entonces surge el problema de como ejecutar un warp en lockstep cuando algunos threads
correran la rama \texttt{true} y otros la rama \texttt{false}. La soluci\'on que adopta CUDA es la
serializaci\'on implicita. Los threads que no ejecutan el \texttt{true} correr\'an \texttt{NOP}
y lo mismo se hara en el caso del \texttt{false}, al reves.

Esto trae aparejado una penalidad importante. Si esa bifurcacion contiene mucho c\'odigo no trivial,
entonces es evidente que se subutilizan importantemente los recursos disponibles.

Habiendo varios de estos casos en los kernels presentes, decidimos utilizar una t\'ecnica sugerida
por NVIDIA. Esta consiste en hacer las operaciones normalmente, como si todos los threads cumplieran
las condiciones del condicional y multiplicar por 1 o por 0 el resultado antes de acumular.
Esto hace que las cuentas que no se ejecutaban antes ahora lo hagan pero que simplemente no aporten
a la reducci\'on. Esta t\'ecnica elimina la existencia de la rama falsa de los condicionales, pero
trae dos problemas no triviales.

Uno de ellos es como solucionar el problema de los accesos a memoria.
Usualmente las guardas condicionales estan puestas para que el programa, si cumple ciertas condiciones,
no acceda a memoria que esta mas alla de los limites definidos. Por ejemplo, si \texttt{threadId > arraySize}, entonces claramente
no deberia acceder a \texttt{array[threadId]}, puesto que seria memoria invalida. Si eliminamos la guarda
entonces los accesos a memoria no pueden quedar igual. Una soluci\'on consiste en multiplicar tambien
por 1 o por 0 la direccion a la cual se va a acceder. Como CUDA maneja los arreglos de memoria
como C (es decir, basados en 0 como primer direcci\'on), esta t\'ecnica es v\'alida para hacer
siempre accesos correcto a memoria. El problema luego es en la coalescencia; como ahora algunos
threads de un warp van a acceder a una posicion de memoria muy distinta a otros, entonces el procesado
va a partir esos accesos en m\'ultiples transacciones. Esto puede hacer que la cuenta no solo no mejore
la performance, sino que puede que la empeore sustancialmente. Se debe hacer un profiling caso
por caso para poder estudiar el impacto en el kernel.

El otro problema es en cuentas que pueden dar NaN (como el tipico caso de division por cero).
Como, por estandar IEEE 754, los NaN hacen que todas las operaciones con ellos den NaN,
entonces pueden propagarse por la cuenta, incluso en con multiplicaci\'on con cero del resultado.
La soluci\'on m\'as evidante seria comprobar si son NaN antes de reducirlas, y si lo fueran, reemplazarlos
por cero. Esto puede llegar a ser inevitable en muchos casos; en el nuestro, con replantear las cuentas,
podemos evitarlos.

\subsection{Escalando m\'as all\'a de un GPU}
Una vez que fueron solucionados muchos limitantes de performance en los kernels del computo
de la densidad electr\'onica y del c\'alculo de la matriz RMM, nos encontramos en un punto donde
no fue posible determinar mejoras significativas en el c\'alculo para reducir tiempos.
Decidimos subir un nivel m\'as el paralelismo, de modo de poder solucionar multiples particiones
simultaneamente. Dado que es independiente el computo de cada partici\'on (salvo la acumulaci\'on
en la matriz RMM de salida y en la matriz de fuerzas interat\'omicas), nos pareci\'o que seria
interesante ver como escala distribuir el computo a lo largo de multiples GPU.

Para dividir el problema entre varios dispositivos usamos, al igual que en CPU, OpenMP. Definimos
una seccion paralela dentro del loop principal donde se soluciona cada grupo de modo que se
ejecutaran tantos threads como placas haya en la maquina. Cada
uno de los threads en el host se configurar\'a para una placa solamente. Esto se realiza con
una instrucci\'on del driver de CUDA (CudaSetDevice) que permite que durante toda la vida del
thread, todas las llamadas a kernels se realicen automaticamente al mismo dispositivo.

CUDA permite que trabajar con multiples placas de esta manera sea bastante sencillo. Las variables
definidas como \texttt{\_\_device\_\_}, que residen plenamente en la GPU, son automaticamente instanciadas
por cada dispositivo presente. De esta manera, es implicito cual variable usa cada kernel; la que
esta definida para su dispositivo actual. Esto puede ser un problema si queremos lograr comunicaciones entre placas,
pero si las cuentas son independientes, es paralelismo gratuito de costo cero.

El principal problema que surge de uso de m\'ultiples dispositivos radica, al igual que en
CPU, en como distribuir la carga de los threads de modo tal que haya una cantidad de trabajo
similar, para minimizar los tiempos de idle. Este problema no es tan grave siempre y cuando
se utilicen placas identicas dentro de la configuraci\'on del sistema ya que seria
el mismo tiempo si se corre en una o en otra. Un trabajo adicional de inter\'es ac\'a
radicaria en el uso de t\'ecnicas de estimacion de poder de computo para poder
distribuir el trabajo de manera equitativa entre modelos de placas heterog\'eneas, con distintas
configuraci\'ones de memoria, cantidad de SM y anchos de banda.

Para distribuir las tareas utilizamos dos t\'ecnicas combinadas, una para distribuir las
tareas estaticamente y otra para redistribuirlas dinamicamente de acuerdo al runtime de
cada tarea.


