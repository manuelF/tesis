% Plan de tesis de licenciatura Diciembre 2014.
% Juan Pablo Darago y Manuel Ferreria

\documentclass[a4paper, 12pt]{article}
\usepackage[spanish]{babel}
%\usepackage{makeidx}
\usepackage[pdftex]{graphicx}

\addtolength{\topmargin}{-2cm}
\addtolength{\textheight}{4cm}
\makeindex

\begin{document}
\DeclareGraphicsExtensions{.jpg,.pdf,.mps,.png}

\title {
\begin{figure}[h]
\begin{center}
\includegraphics[keepaspectratio, width=3cm]{escudo}
\end{center}
\end{figure}
Universidad de Buenos Aires\\
Facultad de Ciencias Exactas y Naturales\\
Departamento de Computaci\'on\\
\vspace{10mm}
\Huge\textbf{Optimizaci\'on de c\'omputo QM/MM empleando arquitecturas masivamente paralelas}
\vspace{10mm}
}

\author{por \\
    Juan Pablo Darago y Manuel Ferrer\'ia\\\
        \\
        Director de Tesis\\
        Dr. Mariano Camilo Gonz\'alez Lebrero\\
        Co-Director de Tesis\\
        Dr. Esteban Mocskos\\
        \\
        Propuesta de tesis para optar al grado de \\
        Licenciado en Ciencias de la Computaci\'on\\
        \\
        }
\date{Diciembre de 2014}

\maketitle

\pagebreak

\section*{Propuesta}

Estudiar y comparar las prestaciones para c\'omputo de alto rendimiento de arquitecturas de procesadores modernos como lo son CPU \textit{multicore}
de la familia Intel IA-64, GPGPU \textit{(General Purpose Graphics Processing Unit)} de NVIDIA y aceleradores Xeon-Phi de Intel. El caso de
estudio para realizar este an\'alisis es una aplicaci\'on del computo de estructuras electr\'onicas usando Teor\'ia del Funcional de la Densidad (DFT) llamado LIO.
Este programa ya ha sido mejorado para utilizar placas GPGPU para generaciones anteriores de NVIDIA.

Las caracter\'isticas del problema y las arquitecturas a estudiar motivan el desarrollo de algoritmos especiales para maximizar el uso de los
recursos disponibles. La implementaci\'on y an\'alisis resultantes de este estudio servir\'an como guia para aplicar optimizaciones a otros
desarrollos en computo de alto rendimiento, y para hacer una decisi\'on informada sobre que arquitectura resulta \'util para que tipos de problema
(al hacerse una comparaci\'on m\'as realista que la que ser\'ia posible utilizando \textit{microbenchmarks}).

\section*{Objetivos}

\begin{itemize}
\item Mejorar la velocidad de ejecuci\'on de las iteraciones de LIO en un 3x en GPGPUs.

\item Mejorar la velocidad de ejecuci\'on de las iteraciones de LIO en un 5x en CPUs.

\item Implementar LIO para arquitectura Xeon Phi.

\item Destilar gu\'ias de optimizaci\'on para las distintas arquitecturas a partir de las mejoras realizadas.

\item Dar una comparaci\'on entre GPGPUs, CPU \textit{multicore} y Xeon Phi seg\'un instancia de problema.
\end{itemize}

\section*{Plan inicial}

\begin{enumerate}

\item Estudiar cuellos de botella de la aplicaci\'on LIO y oportunidades de paralelizaci\'on para uso de m\'ultiples procesadores SMP.

\item Adaptar el c\'odigo existente para nuevas arquitecturas GPGPU (Fermi, Kepler) de NVIDIA.

\item Adaptar el c\'odigo existente para utilizar librer\'ias est\'andar de paralelizaci\'on (OpenMP) y modelos SIMD modernos (AVX, AVX2).

\item Implementar una versi\'on de la aplicaci\'on para Xeon-Phi para hacer un an\'alisis de preliminar de la arquitectura.

\item Comparar calidad num\'erica de las aplicaciones resultantes.

\item Utilizar el programa modificado a tama\~nos de problemas m\'as realistas y previamente inabordables.

\end{enumerate}

\section*{Bibliograf\'ia}

\begin{thebibliography}{9}

\bibitem{LIO}Matias A. Nitsche, Manuel Ferreria, Esteban E. Mocskos, and Mariano C.
Gonz\'alez Lebrero. GPU Accelerated Implementation of Density Functional Theory
for Hybrid QM/MM Simulations. Journal of Chemical Theory and Computation,
10(3):959–967, 2014.

\bibitem{lebrero2002} M. C. Gonz\'alez Lebrero, D. E. Bikiel, M. D. Elola, D. A. Estrin y A. E. Roitberg,
Solvent-induced symmetry breaking of nitrate ion in aqueous clusters: A quantum-classical simulation study.

\bibitem{Jeffers} James Jeffers and James Reinders. Intel Xeon Phi Coprocessor High Performance
Programming. Newnes, 2013.

\bibitem{KS}W. Kohn and L. J. Sham. Self-consistent equations including exchange and correlation
effects. Phys. Rev., 140:A1133–A1138, Nov 1965.

\bibitem{Pacheco}Peter S. Pacheco. An Introduction to Parallel Programming. Morgan Kaufmann
Publishers Inc., 30 Corporate Drive, Suite 400, Burlington, MA 01803, USA, 2011.

\bibitem{CUDAHandbook} Nicholas Wilt. The CUDA Handbook: A comprehensive guide to GPU Programming.
Pearson Education, 2013.

\bibitem{Farber} Rob Farber. CUDA application design and development. Elsevier, 2011.

\bibitem{Amdahl} G.M. Amdahl, Validity of the single-processor approach to achieving large scale computing capabilities(1967)

\bibitem {Gustafson88}John L. Gustafson, Reevaluating Amdahl's Law.(1988)

\end{thebibliography}


\end{document}
