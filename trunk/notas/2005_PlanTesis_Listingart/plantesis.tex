% Plan de tesis de licienciatura Diciembre 2005.
% Pablo Listingart

\documentclass[a4paper, 12pt]{article}
\usepackage[spanish]{babel}
%\usepackage{makeidx}
\usepackage[pdftex]{graphicx}

\addtolength{\topmargin}{-2cm}
\addtolength{\textheight}{4cm}
\makeindex

\begin{document}
\DeclareGraphicsExtensions{.jpg,.pdf,.mps,.png}

\title {
\begin{figure}[h]
\begin{center}
\includegraphics[keepaspectratio, width=3cm]{escudo}
\end{center}
\end{figure}
Universidad de Buenos Aires\\
Facultad de Ciencias Exactas y Naturales\\
Departamento de Computaci\'on\\
\vspace{10mm}
\Huge\textbf{Computaci\'on de alto rendimiento en el estudio de reacciones qu\'imicas en soluci\'on
a trav\'es de m\'etodos cl\'asicos y cu\'anticos.}
\vspace{10mm}
}

\author{por \\
    Pablo Listingart\\
        \\
        Director de Tesis\\
        Dr. Guillermo Marshall\\
        \\
        Propuesta de tesis para optar al grado de \\
        Licenciado en Ciencias de la Computaci\'on\\
        \\
        }
\date{Diciembre de 2005}

\maketitle

\pagebreak

\section*{Propuesta}

Construcci\'on de un software paralelo de simulaci\'on num\'erica a partir de una versi\'on serial preexistente y su implementaci\'on en un cluster Beowulf para el estudio y predicci\'on de reacciones qu\'imicas en soluci\'on.

La combinaci\'on del hardware y software mencionado y el an\'alisis de distintas estrategias de paralelizaci\'on (incluyendo diversas modalidades de pasaje de mensajes) permitir\'an la resoluci\'on de problemas m\'as realistas y previamente inabordables, obteniendo un algoritmo escalable.


\section*{Objetivos}

\begin{itemize}
\item Desarrollar un programa paralelo que resuelva num\'ericamente un esquema h\'ibrido cu\'antico-cl\'asico (QM-MM) en el cual el solvente es tratado cl\'asicamente.

\item Comparar los resultados num\'ericos obtenidos frente a los obtenidos en forma serial del mismo problema.

\item Incorporar herramientas de visualizaci\'on para realizar el an\'alisis de los datos obtenidos.

\item Aplicar el programa obtenido a tama\~nos de problemas m\'as realistas y previamente inabordables.
\end{itemize}

\section*{Elementos a utilizar}

\begin{itemize}
\item Sistema Operativo: Linux, Windows.

\item Lenguaje de programaci\'on: Fortran 77.

\item M\'aquina con suficiente memoria y poder de c\'alculo para poder realizar simulaciones que puedan aproximarse a la realidad f\'isica.

\item C\'odigo fuente de la versi\'on serial que soluciona el modelo.
\end{itemize}

\section*{Plan inicial}

\begin{enumerate}
\item Interiorizarse de la realidad qu\'imica que ocurre durante un experimento en soluci\'on a trav\'es de m\'etodos cl\'asicos y cu\'anticos.

\item Analizar el programa serial utilizado para la resoluci\'on de sistemas qu\'imicos.

\item Determinar las t\'ecnicas posibles de paralelizaci\'on.

\item Realizar la paralelizaci\'on del programa serial siguiendo una estrategia de caja negra.

\item Realizar experimentos para el estudio de reacciones qu\'imicas en soluci\'on a trav\'es de m\'etodos cl\'asicos y cu\'anticos.

\item Analizar los resultados y relizar pruebas de performance.

\end{enumerate}

\section*{Bibliograf\'ia}

\begin{thebibliography}{9}

\bibitem {Nemukhin2002}A. V. Nemukhin, I. A. Topol, B. L. Grigorenko y S. K. Burt, On the Origin of Potential Barrier for the Reaction OH$^-$ + CO$_2$ $\rightarrow$ HCO$_3$ in Water: Studies by Using Continuum and Cluster Solvation
Methods(2002)

\bibitem {Kesavan2003} L. S. Devi-Kesavan y J. Gao, Combined QM/MM Study of the Mechanism and Kinetic Isotope Effect
of the Nucleophilic Substitution Reaction in Haloalkane
Dehalogenase (2003)

\bibitem {lebrero2002} M. C. Gonz\'alez Lebrero, D. E.
Bikiel, M. D. Elola, D. A. Estrin y A. E. Roitberg,
Solvent-induced symmetry breaking of nitrate ion in aqueous
clusters: A quantum-classical simulation study

\bibitem {mpich-install}William D. Gropp and Ewing Lusk, Installation Guide for {\tt mpich}, a Portable Implementation of
{MPI}(1996)

\bibitem {LAM_MPI}Local Area Multicomputer - MPI Parallel Computing Environment, http://www.osc.edu/lam.html

\bibitem {MPICH}MPICH(A Portable MPI Implementation), http://www.mcs.anl.gov/mpi/mpich

\bibitem {Pacheco96}Peter S. Pacheco, Parallel programming with MPI(1996)

\bibitem {Gropp94}William Gropp, Ewing Lusk y Anthony Skjellum, Using MPI: Portable Parallel Programming with the Message-Passing
Interface(1994)

\bibitem {LSC}Laboratorio de Sistemas Complejos, Departamento de Computaci\'on, Facultad de Ciencias Exactas y Naturales, Universidad de Buenos Aires, http://lsc.dc.uba.ar

\bibitem {Amdahl}G.M. Amdahl, Validity of the single-processor approach to achieving large scale computing capabilities(1967)

\bibitem {Gustafson88}John L. Gustafson, Reevaluating Amdahl's Law.(1988)

\end{thebibliography}


\end{document}
